\documentclass[12pt,a4paper]{article}
\usepackage{amssymb,amsmath}
\usepackage[dvips]{lscape,graphicx}

\voffset=0mm
\hoffset=0mm
\oddsidemargin=0mm
\textwidth=160mm
\headheight=0mm
\topmargin=0mm
\headsep=0mm
\textheight=246mm
\footskip=13mm
\binoppenalty=10000
\relpenalty=10000

%\parindent=0mm

\renewcommand{\thesection}{\arabic{section}.}
\renewcommand{\thesubsection}{\arabic{section}.\arabic{subsection}.}
\renewcommand{\baselinestretch}{1.30}
\newcommand{\cen}{\centerline}
\newcommand{\bc}{\begin{center}}
\newcommand{\ec}{\end{center}}
\newcommand{\bd}{\begin{displaymath}}
\newcommand{\ed}{\end{displaymath}}
\newcommand{\be}{\begin{equation}}
\newcommand{\ee}{\end{equation}}
\newcommand{\ba}{\begin{array}}
\newcommand{\ea}{\end{array}}
\newcommand{\bt}{\begin{tabular}}
\newcommand{\et}{\end{tabular}}
\newcommand{\un}{\underline}
\newcommand{\ov}{\overline}
\newcommand{\ds}{\displaystyle}
\newcommand{\rre}{\mathrm{Re}}
\newcommand{\iim}{\mathrm{Im}}

\sloppy


\begin{document}

\title{
%\hfill
\textbf{Nonstandard Higgs decays\\ in the $U(1)$ extension of the MSSM\\[4mm]}}

\date{}
\author{
P.~Athron${}^{a}$,
M.~M\"{u}hlleitner$^{b}$,
R.~Nevzorov${}^{a}$\footnote{On leave of absence from the Theory Department, SSC RF ITEP of NRC "Kurchatov Institute", Moscow, Russia.},
A.~G.~Williams${}^{a}$\\[9mm]
{\small\it $^a$ ARC Centre of Excellence for Particle Physics at the Terascale and CSSM,}\\
{\small\it School of Chemistry and Physics, The University of Adelaide, Adelaide SA 5005, Australia}\\[3mm]
{\small\it $^b$ Institute for Theoretical Physics, Karlsruhe Institute of Technology,} \\
{\small\it 76128 Karlsruhe, Germany.}\\
}

\maketitle

\begin{abstract}
\noindent
In $U(1)$ extensions of the minimal supersymmetric (SUSY) standard
model (MSSM) there is a simple mechanism that leads to a heavy $Z'$
with a mass which is substantially larger than the SUSY breaking
scale. This mechanism may also result in a pseudoscalar state so light
that the decays of the $125\,\mbox{GeV}$ SM-like Higgs boson into a
pair of such pseudoscalars can be kinematically allowed.  We study
these decays within $E_6$ inspired SUSY models with exact custodial
symmetry that forbids tree-level flavor-changing transitions and the
most dangerous baryon and lepton number violating operators. We argue
that the branching ratio of the lightest Higgs boson decays into a
pair of the light pseudoscalar states may not be negligibly small.
\end{abstract}
\thispagestyle{empty}
\vfill
\newpage
\setcounter{page}{1}



\section{Introduction}
Searches for physics beyond the standard model (SM) with ATLAS and CMS
have set rather stringent constraints on the masses of sparticles and
$Z'$ bosons. Indeed, in the case of the $E_6$ inspired models the LHC
data analysis excludes $Z'$ resonances with masses $M_{Z'}$ below
$2.5\,\mbox{TeV}$ \cite{1,2}. In the simplest $U(1)$ extensions of the
minimal supersymmetric (SUSY) standard model (MSSM) the extra $U(1)$
gauge symmetry is normally broken by the vacuum expectation value
(VEV) of the SM singlet superfield $S$ that carries non--zero $U(1)$
charges. Since $\langle S\rangle$ and $M_{Z'}$ are determined by the
SUSY breaking scale in these models the multi--TeV $Z'$ mass typically
implies that other sparticles also have multi--TeV masses.

 Since the third generation sfermions and Higgs bosons with sub–-TeV
 masses have not been ruled out by the LHC experiments yet it is worth
 to consider the phenomenological implications of the $U(1)$
 extensions of the MSSM in which the $Z'$ boson can be substantially
 heavier than the sparticles. Such scenarios may be realised when the
 extra $U(1)$ gauge symmetry is broken by two VEVs of the SM singlet
 superfields $S$ and $\overline{S}$ with opposite $U(1)$ charges.  In
 this case the extra $U(1)$ D--term contribution to the scalar
 potential may force the minimum of this potential to be along the
 $D$--flat direction (see, for example, \cite{Kolda:1995iw}).  As a
 consequence $\langle S\rangle$ and $\langle\overline{S}\rangle$ can
 be much larger than the SUSY breaking scale.

The simplest renormalizable superpotential of the SUSY model of the type discussed above
can be written as
\be
W_S=\sigma \phi S \overline{S}\,,
\label{hd1}
\ee
where $\phi$ is a scalar superfield that does not participate in gauge interactions. When
$\sigma$ goes to zero the corresponding tree--level scalar potential takes a form
\begin{equation}
V_S = m^2_S |S|^2 + m^2_{\overline{S}} |\overline{S}|^2 + m^2_{\phi} |\phi|^2
+\ds\frac{Q_S^2 g^{'2}_1}{2}\left(|S|^2-|\overline{S}|^2\right)^2\,,
\label{hd2}
\end{equation}
where $m_S^2$, $m^2_{\overline{S}}$ and $m^2_{\phi}$ are soft SUSY breaking
parameters and $Q_S$ is a $U(1)'$ charge of the SM singlet superfields $S$. In Eq.~(\ref{hd2})
the last term is associated with the extra $U(1)$ D--term contribution. In the limit
$\langle S \rangle = \langle \overline{S} \rangle$ this quartic term vanishes. If
$(m^2_S + m^2_{\overline{S}})<0$ then there is a run--away direction in this
model, so that $\langle S \rangle = \langle \overline{S} \rangle \to \infty$.
For small values $\sigma$ the interaction term in the superpotential (\ref{hd1}) stabilizes
the run-away direction and the SM singlet superfields tend to acquire large VEVs,
i.e
\be
\langle \phi \rangle \sim \langle S \rangle \simeq \langle \overline{S} \rangle
\sim \dfrac{1}{\sigma}\sqrt{|m^2_S + m^2_{\overline{S}}|}\,,
\label{hd3}
\ee
resulting in an extremely heavy $Z'$ boson.

Although the SUSY model mentioned above looks rather simple and elegant it
possess an additional accidental global $U(1)$ symmetry which can be
associated with the Peccei-Quinn (PQ) symmetry \cite{Peccei:1977hh}.
This symmetry is spontaneously broken by the VEVs of the SM singlet superfields
resulting in massless axion \cite{axion}. To avoid the appearance of axion
one needs to include in the superpotential (\ref{hd1}) polynomial terms with respect
to the superfield $\phi$ which explicitly break the global $U(1)$ symmetry. If the PQ
symmetry violating couplings are very small then the particle spectrum of this
SUSY model should contain a pseudo--Goldstone boson which can be considerably
lighter than all sparticles and Higgs bosons. In fact, the corresponding pseudoscalar
Higgs state may be so light that the SM-like Higgs boson decays into a pair of
these states can be kinematically allowed.

In this article we consider such nonstandard Higgs decays within the
well motivated $E_6$ inspired extensions of the MSSM. The breakdown of
the $E_6$ symmetry at high energies may lead to the models based on
rank--5 gauge groups with additional $U(1)'$ factor in comparison to
the SM. In this case extra $U(1)'$ gauge symmetry is a linear
combination of $U(1)_{\chi}$ and $U(1)_{\psi}$: \be
U(1)'=U(1)_{\chi}\cos\theta+U(1)_{\psi}\sin\theta\,,
\label{hd4}
\ee
which are defined by: $E_6\to SO(10)\times U(1)_{\psi},~SO(10)\to SU(5)\times U(1)_{\chi}$
(for a review see \cite{Langacker:2008yv}--\cite{E6-review}).  With additional Abelian gauge
symmetries it is important to ensure the cancellation of anomalies. In any model based on the
subgroup of $E_6$ anomalies are cancelled automatically if the low energy spectrum involves
complete $27$-plets. Consequently, in $E_6$ inspired SUSY models the particle spectrum is
extended to fill out three complete 27-dimensional representations of $E_6$. Each $27$--plet
contains one generation of ordinary matter; SM singlet field, $S_i$, that carries non--zero $U(1)'$
charge; up and down type Higgs doublets, $H^{u}_{i}$ and $H^{d}_{i}$; charged
$\pm 1/3$ exotic quarks $D_i$ and $\bar{D}_i$.

Different aspects of the phenomenology of the $E_6$ inspired SUSY models have been
extensively studied over the years \cite{E6-review}--\cite{E6}. A few years ago the Tevatron
and early LHC $Z'$ mass limits in these models have been discussed in \cite{Accomando:2010fz}
and collider signatures associated with the exotic quarks and squarks have been considered
in \cite{Kang:2007ib}. Previously, the implications of $E_6$ inspired SUSY models with an
additional $U(1)'$ gauge symmetry have been studied for electroweak (EW) symmetry breaking
(EWSB) \cite{Langacker:1998tc}--\cite{Daikoku:2000ep}, neutrino physics \cite{Kang:2004ix}--\cite{Ma:1995xk},
leptogenesis \cite{Hambye:2000bn}--\cite{King:2008qb},
EW baryogenesis \cite{baryogen}, muon anomalous magnetic moment \cite{g-2}, electric
dipole moment of electron \cite{Suematsu:1997tv} and tau lepton \cite{GutierrezRodriguez:2006hb},
lepton flavour violating processes like $\mu\to e\gamma$ \cite{Suematsu:1997qt} and CP-violation in the
Higgs sector \cite{Ham:2008fx}. The neutralino sector in $E_6$ inspired SUSY models was analysed
in \cite{Keith:1997zb}, \cite{Suematsu:1997tv}--\cite{Suematsu:1997qt},
\cite{Suematsu:1997au}--\cite{E6neutralino-higgs}.  Such models have also been proposed as
the solution to the tachyon problems of anomaly mediated SUSY breaking, via $U(1)^\prime$ D-term
contributions \cite{Asano:2008ju}, and used in combination with a generation symmetry to construct
a model explaining fermion mass hierarchy and mixing \cite{Stech:2008wd}. The Higgs sector and
the theoretical upper bound on the lightest Higgs boson mass in the $E_6$ inspired SUSY models
were examined in \cite{Daikoku:2000ep}, \cite{E6neutralino-higgs}, \cite{King:2005jy}--\cite{E6-higgs}.

Here we focus on the $E_6$ inspired SUSY extension of the SM based on
the low--energy SM gauge group together with an extra $U(1)_{N}$ gauge
symmetry in which right--handed neutrinos do not participate in the
gauge interactions. This corresponds to $\theta=\arctan\sqrt{15}$.
Only in this Exceptional Supersymmetric Standard Model (E$_6$SSM)
\cite{King:2005jy}--\cite{King:2005my} right--handed neutrinos may be
superheavy, shedding light on the origin of the mass hierarchy in the
lepton sector and providing a mechanism for the generation of the
baryon asymmetry in the Universe via leptogenesis
\cite{Hambye:2000bn}--\cite{King:2008qb}. $E_6$ inspired SUSY models
with an additional $U(1)_{N}$ gauge symmetry have been studied in
\cite{Ma:1995xk} in the context of non--standard neutrino models with
extra singlets, in \cite{Suematsu:1997au} from the point of view of
$Z-Z'$ mixing, in \cite{Keith:1997zb} and
\cite{Suematsu:1997au}--\cite{Keith:1996fv} where the neutralino
sector was explored, in \cite{Keith:1997zb}, \cite{King:2007uj} where
the renormalisation group (RG) flow of couplings was examined and in
\cite{Suematsu:1994qm}--\cite{Daikoku:2000ep} where EWSB was
studied. The presence of a $Z'$ boson and of exotic quarks predicted
by the E$_6$SSM provides spectacular new physics signals at the LHC
which were analysed in \cite{King:2005jy}--\cite{Accomando:2006ga},
\cite{Howl:2007zi}. The presence of light exotic particles also lead
to the nonstandard decays of the SM--like Higgs boson that were
discussed in detail in \cite{Nevzorov:2013tta}--\cite{Hall:2010ix}.
Recently the particle spectrum and collider signatures associated with
it were studied within the constrained version of the E$_6$SSM
\cite{cE6SSM}. The degree of fine tuning in the constrained version has has been examined recently \cite{Athron:2013ipa}. The threshold corrections to the running gauge and
Yukawa couplings in the E$_6$SSM and cE$_6$SSM were studied in detail
in \cite{Athron:2012pw}. Alternative boundary conditions which are shifted by D-terms from the highscale breakdown of the U(1) gauge symmetry which doesn't survive to low energies have been considered in the context of using a measuremet of the first or second generation sefermion masses to constrain the GUT scale parameters\cite{Miller:2012vn}. The renormalization of VEVs in the E$_6$SSM
was considered in \cite{Sperling:2013eva}.

The presence of exotic matter in the E$_6$SSM generically lead to non--diagonal flavour transitions and rapid proton
decay. To suppress these processes one can impose a set of discrete symmetries \cite{King:2005jy}--\cite{King:2005my}.
In this article we study the nonstandard Higgs decays mentioned above within the $E_6$ inspired SUSY models
with the extra $U(1)_{N}$ factor in which a single discrete $\tilde{Z}^{H}_2$ symmetry forbids tree--level
flavor--changing transitions and the most dangerous baryon and lepton number violating operators \cite{nevzorov}.
In particular, we consider scenarios with approximate global $U(1)$ symmetry that leads to a pseudo--Goldstone
boson in the particle spectrum. Since no indication for the presence of SUSY particles has been observed we
assume that the SUSY breaking parameters are of the order of $1\,\mbox{TeV}$ or even larger. The pseudo--Goldstone
state in these scenarios is mainly a linear superposition of the imaginary parts of the scalar components of the
SM singlet superfields $\phi$, $S$ and $\overline{S}$. Although one can expect that the coupling of this state
to the SM--like Higgs boson, which is predominantly a linear superposition of the neutral components of the Higgs
doublets $H_u$ and $H_{d}$, is somewhat suppressed it can still lead to the non--negligible branching ratio of the
lightest Higgs decays into a pair of pseudo--Goldstone bosons.

In this context it is worth noting that the decay rate of the SM-like Higgs state into a pair of pseudoscalars were
intensively studied within the simplest extension of the MSSM ---  Next--to--Minimal Supersymmetric Standard Model
(NMSSM) (for the reviews of nonstandard Higgs boson decays see \cite{Chang:2008cw}). The NMSSM superpotential
is given by \cite{review-nmssm}:
\begin{equation}
W_{NMSSM}=\lambda S(H_d H_u)+\dfrac{\kappa}{3} S^3 + W_{MSSM}(\mu=0)\,.
\label{hd5}
\end{equation}
In the limit $\kappa= 0$ the Lagrangian of the NMSSM is invariant under the transformations of the
PQ symmetry that leads to the massless axion when it is spontaneously broken by the VEV $\langle S\rangle$.
If $\kappa$ is rather small then the NMSSM particle spectrum involves one light scalar and one light pseudoscalar
states. In addition if $\kappa\to 0$ and the SUSY breaking scale is of the order of $\mbox{TeV}$
the lightest SUSY particle (LSP) is the lightest neutralino $\tilde{\chi}^0_1$ which is predominantly singlino.
In this case the LSP couplings to the SM particles are quite small resulting in a relatively small annihilation cross
section for $\tilde{\chi}^0_1\tilde{\chi}^0_1\to \mbox{SM particles}$ which gives rise to a relic density that is
typically much larger than its measured value. As a consequence it seems to be rather problematic to find
phenomenologically viable scenarios with light pseudoscalar in the case of the NMSSM with approximate PQ symmetry.
Nevertheless the sufficiently light pseudoscalar can be always obtained by tuning the parameters of the NMSSM.

In contrast with the NMSSM the mass of the lightest neutralino in the
SUSY model considered here does not become small when the PQ symmetry
violating couplings vanish. Moreover, even when all PQ symmetry
violating couplings are negligibly small, the LSP can be
higgsino--like, which allows a reasonable value for the dark matter
density to be obtained if LSP has a mass below $1\,\mbox{TeV}$ (see,
for example \cite{ArkaniHamed:2006mb}). Thus the approximate PQ
symmetry can lead to phenomenologically viable scenarios with light
pseudoscalar in this model.

The paper is organised as follows. In the next section we briefly review the $E_6$ inspired SUSY models with exact
custodial $\tilde{Z}^{H}_2$ symmetry. In section 3 the breakdown of gauge symmetry and Higgs phenomenology
are studied. In section 4 we discuss a set of benchmark scenarios that leads to the decays of the lightest Higgs boson
into a pair of pseudoscalar states. Our results are summarised in section 5. In Appendix A the spectrum of neutralino
states is examined.



\section{$E_6$ inspired SUSY models with exact $\tilde{Z}^{H}_2$ symmetry}

In this section, we briefly review the $E_6$ inspired SUSY models with
exact custodial $\tilde{Z}^{H}_2$ symmetry \cite{nevzorov}, which we
use to demonstrate how light pseudoscalar states can appear in SUSY
models with an extra $U(1)'$ gauge symmetry and what kind of Higgs
decay rates they can lead to.  These models imply that $E_6$ or its
subgroup is broken down to $SU(3)_C\times SU(2)_W\times U(1)_Y\times
U(1)_{\psi}\times U(1)_{\chi}$ near the Grand Unification scale $M_X$
(GUT scale). Below scale $M_X$ the particle content of these SUSY
models involves three copies of $27_i$--plets and a set of $M_{l}$ and
$\overline{M}_l$ supermultiplets from the incomplete $27'_l$ and
$\overline{27'}_l$ representations of $E_6$. All components of the
complete $27_i$--plets are odd under $\tilde{Z}^{H}_2$ discrete
symmetry, while the supermultiplets $\overline{M}_l$ can be either odd
or even. The supermultiplets $M_{l}$ are even under the
$\tilde{Z}^{H}_2$ symmetry and as a consequence they can be used for
the breakdown of gauge symmetry. To ensure that the $SU(2)_W\times
U(1)_Y\times U(1)_{\psi}\times U(1)_{\chi}$ symmetry is broken down to
$U(1)_{em}$ associated with the electromagnetism the set of multiplets
$M_{l}$ should involve $H_u$, $H_d$, $S$ and $N^c_H$.  Just below the
GUT scale $U(1)_{\psi}\times U(1)_{\chi}$ gauge symmetry is expected
to be broken by the VEVs of $N^c_H$ and $\overline{N}_H^c$ down to
$U(1)_{N}\times Z_{2}^{M}$ in these $E_6$ inspired models, where
$Z_{2}^{M}=(-1)^{3(B-L)}$ is a matter parity. This is possible because
matter parity is a discrete subgroup of $U(1)_{\psi}$ and
$U(1)_{\chi}$. Due to such breakdown of $U(1)_{\psi}$ and
$U(1)_{\chi}$ gauge symmetries all exotic states which originate from
$27_i$ representations of $E_6$ as well as ordinary quark and lepton
states survive to low energies. In general the large VEVs $\langle
N^c_H \rangle \sim \langle \overline{N}_H^c \rangle \lesssim M_X$ also
induce the large Majorana masses for right-handed neutrinos allowing
them to be used for the see--saw mechanism.  Since $N^c_H$ and
$\overline{N}_H^c$ acquire VEVs both supermultiplets should be even
under the imposed $\tilde{Z}^{H}_2$ symmetry.

At the TeV scale the superfields $H_u$, $H_d$ and $S$ play the role of
Higgs fields. The VEVs of these superfields break the $SU(2)_W\times
U(1)_Y\times U(1)_{N}$ gauge symmetry down to $U(1)_{em}$.  Because of
this the supermultiplets $H_u$, $H_d$ and $S$ should be also even
under the $\tilde{Z}^{H}_2$ symmetry.  In contrast, in the simplest
scenario $\overline{H}_u$ and $\overline{H}_d$ are expected to be odd
under this custodial symmetry so that they can get combined with the
superposition of the corresponding components from the $27_i$, forming
vectorlike states that gain masses of order of $M_X$.  The superfield
$\overline{S}$ may also acquire non--zero VEV breaking $U(1)_{N}$
symmetry. If this is a case then $\overline{S}$ has to be even under
the $\tilde{Z}^{H}_2$ symmetry. When $\overline{S}$ is odd under the
$\tilde{Z}^{H}_2$ symmetry then it can get combined with the
superposition of the appropriate components of $27_i$ resulting in
formation of superheavy vectorlike states with masses $\sim M_X$.  The
custodial $\tilde{Z}^{H}_2$ symmetry allows the Yukawa interactions in
the superpotential that originate from $27'_l \times 27'_m \times
27'_n$ and $27'_l \times 27_i \times 27_k$. It is easy to check that
the corresponding set of operators does not contain any that lead to
the rapid proton decay. Since the set of multiplets $M_{l}$ contains
only one pair of doublets $H_d$ and $H_u$ the down-type quarks and
charged leptons couple to just one Higgs doublet $H_d$, whereas the
up-type quarks couple to $H_u$ only. As a result the flavor--changing
processes are forbidden at the tree level.

Nonetheless if the set of $\tilde{Z}^{H}_2$ even supermultiplets
$M_{l}$ involve only $H_u$, $H_d$, $S$ and $N^c_H$ then the Lagrangian
of the model is invariant not only with respect to $U(1)_B$ but also
under $U(1)_D$ symmetry transformations \be D\to e^{i\alpha}
D\,,\qquad\qquad \overline{D}\to e^{-i\alpha}\overline{D}\,.
\label{hd6}
\ee
The $U(1)_D$ symmetry ensures that the lightest exotic quark is very long--lived.
Indeed, as $U(1)_B$ the $U(1)_D$ global symmetry is expected to be broken by a set of non--renormalizable
operators which are suppressed by inverse power of $M_X$ or $M_{Pl}$. These operators allow the lightest
exotic quark to decay. However its lifetime tends to be considerably larger than the age of the Universe.
So long--lived exotic quarks would have been produced during the very early epochs of the Big Bang.
Those lightest exotic quark states which survive annihilation would subsequently have been confined
in heavy hadrons forming nuclear isotopes that should be present in terrestrial matter. Various theoretical
estimations \cite{43} show that if such stable relics in the mass range from $1\,\mbox{GeV}$ to $10\,\mbox{TeV}$
would exist in nature today their concentration is expected to be at the level of $10^{-10}$ per nucleon.
At the same time different experiments set strong upper limits on the relative concentrations of such
nuclear isotopes from $10^{-15}$ to $10^{-30}$ per nucleon \cite{42}. Therefore $E_6$ inspired models
with very long--lived exotic quarks are ruled out.

To ensure that the lightest exotic quarks decay within a reasonable time in the simplest scenario
the set of $\tilde{Z}^{H}_2$ even supermultiplets $M_{l}$ is supplemented by $L_4$, where $L_4$ and
$\overline{L}_4$ are lepton $SU(2)_W$ doublet supermultiplets that originate from a pair of additional
$27'_{L}$ and $\overline{27'}_L$. The supermultiplets $L_4$ and $\overline{L}_4$ should form TeV scale
vectorlike states to render the lightest exotic quark unstable\footnote{The appropriate mass term
$\mu_L L_4\overline{L}_4$ in the superpotential can be induced within SUGRA models just after the
breakdown of local SUSY if the K\"ahler potential contains an extra term $(Z_L (L_4\overline{L}_4)+h.c)$\cite{45}.}.
Therefore $L_4$ and $\overline{L}_4$ have to be both even under the $\tilde{Z}^{H}_2$ symmetry.
In this case the baryon and lepton number conservation implies that exotic quarks are leptoquarks.

Here we assume that in addition to $H_u$, $H_d$, $S$, $L_4$,
$\overline{L}_4$ $N^c_H$ and $\overline{N}_H^c$ the particle spectrum
below scale $M_X$ involves $\tilde{Z}^{H}_2$ even superfields
$\overline{S}$ and $\phi$, where superfield $\phi$ does not
participate in the $SU(3)_C\times SU(2)_W\times U(1)_Y\times
U(1)_{\psi}\times U(1)_{\chi}$ gauge interactions but acquires
non--zero VEVs. Taking into account that the components of the
superfields $\overline{S}$ and $\phi$ are expected to gain
$\mbox{TeV}$ scale masses whereas the right--handed neutrino
superfields are superheavy, the low energy matter content in the $E_6$
inspired SUSY models discussed above involves: \be \ba{c}
3\left[(Q_i,\,u^c_i,\,d^c_i,\,L_i,\,e^c_i)\right]
+3(D_i,\,\bar{D}_i)+3(S_{i})+2(H^u_{\alpha})+2(H^d_{\alpha})\\[2mm]
+L_4+\overline{L}_4+S+\overline{S}+H_u+H_d+\phi\,, \ea
\label{hd7}
\ee
where $\alpha=1,2$ and $i=1,2,3$. Neglecting all suppressed non-renormalisable interactions
the low-energy effective superpotential of these models can be written as
\be
\ba{rcl}
W &=& \lambda S (H_u H_d) - \sigma \phi S \overline{S} + \dfrac{\kappa}{3}\phi^3+\dfrac{\mu}{2}\phi^2+\Lambda\phi\\[2mm]
&&+ \lambda_{\alpha\beta} S (H^d_{\alpha} H^u_{\beta})+ \kappa_{ij} S (D_{i} \overline{D}_{j}) + \tilde{f}_{i\alpha} S_{i} (H^d_{\alpha} H_u)
+ f_{i\alpha} S_{i} (H_d H^u_{\alpha}) \\[2mm]
&&+ g^D_{ij} (Q_i L_4) \overline{D}_j+ h^E_{i\alpha} e^c_{i} (H^d_{\alpha} L_4) + \mu_L L_4\overline{L}_4+
\tilde{\sigma} \phi L_4\overline{L}_4+ W_{MSSM}(\mu=0)\,.
\ea
\label{hd8}
\ee

The gauge group and field content of the $E_6$ inspired SUSY models under consideration
can originate from the 5D and 6D orbifold GUT models in which the splitting of GUT
multiplets can be naturally achieved \cite{nevzorov}. In these orbifold GUT models
all GUT relations between the Yukawa couplings can get spoiled while the approximate
unification of the SM gauge couplings should take place. From Eq.~(\ref{hd7}) it follows
that extra matter beyond the MSSM fill in complete $SU(5)$ representations in these
models. As a consequence the gauge coupling unification remains almost exact in the
one--loop approximation. It was also shown that in the two--loop approximation the unification
of the gauge couplings in the SUSY models under consideration can be achieved for any
phenomenologically acceptable value of $\alpha_3(M_Z)$, consistent with the central
measured low energy value \cite{nevzorov}, \cite{King:2007uj}.

\begin{table}[ht]
\centering
\begin{tabular}{|c|c|c|c|c|c|c|c|c|c|}
\hline
                   &  $27_i$          &   $27_i$              &$27'_{H_u}$&$27'_{S}$&
$\overline{27'}_{H_u}$&$\overline{27'}_{S}$&$27'_N$&$27'_{L}$&$1$\\
& & &$(27'_{H_d})$& &$(\overline{27'}_{H_d})$& &$(\overline{27'}_N)$&$(\overline{27'}_L)$&\\
\hline
                   &$Q_i,u^c_i,d^c_i,$&$\overline{D}_i,D_i,$  & $H_u$     & $S$     &
$\overline{H}_u$&$\overline{S}$&$N^c_H$&$L_4$&$\phi$\\
                   &$L_i,e^c_i,N^c_i$ &  $H^d_{i},H^u_{i},S_i$& $(H_d)$   &         &
$(\overline{H}_d)$&&$(\overline{N}_H^c)$&$(\overline{L}_4)$&\\
\hline
$\tilde{Z}^{H}_2$  & $-$              & $-$                   & $+$       & $+$     &
$-$&$\pm$&$+$&$+$&$+$\\
\hline
$Z_{2}^{M}$        & $-$              & $+$                   & $+$       & $+$     &
$+$&$+$&$-$&$-$&$+$\\
\hline
$Z_{2}^{E}$        & $+$              & $-$                   & $+$       & $+$     &
$-$&$\pm$&$-$&$-$&$+$\\
\hline
\end{tabular}
\caption{Transformation properties of different components of $E_6$ multiplets
under $\tilde{Z}^H_2$, $Z_{2}^{M}$ and $Z_{2}^{E}$ discrete symmetries.}
\label{tab-hd1}
\end{table}

For the analysis of the phenomenological implications of SUSY models discussed
above it is convenient to introduce $Z_{2}^{E}$ symmetry which is defined so that
$\tilde{Z}^{H}_2 = Z_{2}^{M}\times Z_{2}^{E}$. The transformation properties
of different components of $27_i$, $27'_l$ and $\overline{27'}_l$ supermultiplets under
the $\tilde{Z}^{H}_2$, $Z_{2}^{M}$ and $Z_{2}^{E}$ symmetries are summarized
in Table~\ref{tab-hd1}. Since the low--energy effective Lagrangian of the $E_6$ inspired
SUSY models studied here is invariant under the transformations of $Z_{2}^{M}$ and
$\tilde{Z}^{H}_2$ symmetries, the $Z_{2}^{E}$ symmetry associated with the exotic
states is also conserved. The invariance of the Lagrangian under the $Z_{2}^{E}$ symmetry
implies that in collider experiments the exotic particles, which are odd under this symmetry,
can only be created in pairs and the lightest exotic state should be absolutely stable.
Using the method proposed in \cite{Hesselbach:2007te} it was argued that the masses
of the lightest inert neutralino states\footnote{We use the terminology ``Inert Higgs'' to
denote Higgs--like doublets and SM singlets that do not develop VEVs. The fermionic components
of these supermultiplets form inert neutralino and chargino states.}, which are predominantly linear
superpositions of the fermion components of the superfields $S_i$ from complete $27_i$
representations of $E_6$, do not exceed $60-65\,\mbox{GeV}$ \cite{Hall:2010ix}.
Because of this the corresponding states tend to be the lightest exotic particles in the spectrum.

The presence of lightest exotic particles with masses below $60\,\mbox{GeV}$ gives rise to
new decay channels of the SM--like Higgs boson. Moreover if these states are heavier
than $5-10\,\mbox{GeV}$ then the SM--like Higgs state decays predominantly into the
lightest inert neutralinos while the total branching ratio into SM particles gets strongly suppressed.
Nowadays such scenarios are basically ruled out. On the other hand if the lightest exotic particles
have masses below $5\,\mbox{GeV}$ their couplings to the SM--like Higgs state, gauge bosons,
quarks and leptons are very small. This allows to avoid problems with nonstandard Higgs
decays but it results in the cold dark matter density which is much larger than its measured value
because the corresponding annihilation cross section tends to be rather small.
The simplest phenomenologically viable scenarios imply that the lightest inert neutralinos
are substantially lighter than $1\,\mbox{eV}$\footnote{The presence of very light neutral fermions
in the particle spectrum might have interesting implications for the neutrino physics (see, for
example \cite{Frere:1996gb}).}. This can be achieved if $f_{\alpha\beta}\sim \tilde{f}_{\alpha\beta}\lesssim 10^{-6}$
In this case the lightest exotic particles form hot dark matter (dark radiation) in the Universe
but give only a very minor contribution to the dark matter density.

The $Z_{2}^{M}$ symmetry conservation ensures that $R$--parity is also conserved in the SUSY models
discussed above. Because the lightest exotic state is the lightest $R$--parity odd state either the lightest
$R$--parity even exotic state or the lightest $R$--parity odd state with $Z_{2}^{E}=+1$ must be
stable. In the $E_6$ inspired models studied here most commonly the lightest ordinary neutralino state
($Z_{2}^{E}=+1$) tend to be absolutely stable. As in the MSSM this state may account for all or some
of the observed cold dark matter density.

As mentioned before, in the simplest case the sector responsible for the breakdown of the
$SU(2)_W\times U(1)_Y\times U(1)_{N}$ gauge symmetry involves $H_u$, $H_d$ and $S$. In this case
the Higgs sector of the $E_6$ inspired SUSY models with the extra $U(1)_{N}$ factor was explored in \cite{King:2005jy}.
If CP--invariance is preserved then the Higgs spectrum in these models contains three CP--even, one CP--odd
and two charged states.  The singlet dominated CP--even state is always almost degenerate with the $Z'$ gauge
boson. In contrast with the MSSM, the lightest Higgs boson in these models can be heavier than $110-120\,\mbox{GeV}$
even at tree level. In the two--loop approximation the lightest Higgs boson mass does not exceed $150-155\,\mbox{GeV}$
\cite{King:2005jy}. Recently, the RG flow of the Yukawa couplings and the theoretical upper bound on
the lightest Higgs boson mass in these models were analysed in the vicinity of the quasi--fixed point \cite{Nevzorov:2013ixa}
that appears as a result of the intersection of the invariant and quasi--fixed lines \cite{Nevzorov:2001vj}.
It was argued that near the quasi--fixed point the upper bound on the mass of the SM--like Higgs boson is
rather close to $125\,\mbox{GeV}$ \cite{Nevzorov:2013ixa}.

The qualitative pattern of the Higgs spectrum
in the $E_6$ inspired SUSY models with the extra $U(1)_{N}$ gauge symmetry and minimal Higgs sector
is determined by the Yukawa coupling $\lambda$. When $\lambda < g'_1$, where $g'_1$ is a gauge coupling
associated with the $U(1)_{N}$ gauge symmetry, the singlet dominated CP--even state is very heavy and
decouple from the rest of the spectrum making it indistinguishable from the one in the MSSM. If $\lambda\gtrsim g'_1$
the Higgs spectrum has extremely hierarchical structure which is rather similar to the one that arises in the NMSSM
with the approximate PQ symmetry \cite{Miller:2005qua}--\cite{Nevzorov:2004ge}. As a consequence
the mass matrix of the CP--even Higgs sector can be diagonalized using the perturbation theory
\cite{Nevzorov:2004ge}--\cite{Nevzorov:2001um}. In this case the mass of the second lightest CP--even
Higgs state is set by the $Z'$ boson mass while the heaviest CP--even, CP--odd and charged states
are almost degenerate and lie beyond the multi TeV range.


\section{Higgs sector}
\subsection{The Higgs potential and gauge symmetry breaking}

As it was mentioned in the previous section the sector responsible for the breakdown of gauge symmetry in the SUSY
model under consideration includes two Higgs doublets $H_u$ and $H_d$ as well as the SM singlet fields $S$,
$\overline{S}$ and $\phi$. The interactions between these fields are determined by the structure of the gauge
interactions and by the superpotential in Eq.~(\ref{hd8}). The resulting Higgs effective potential is the sum
of four pieces:
\be
\ba{rcl}
V&=&V_F+V_D+V_{soft}+\Delta V\, ,\\[2mm]
V_F&=&\lambda^2|S|^2(|H_d|^2+|H_u|^2) + \left|\lambda (H_d H_u) - \sigma \phi \overline{S}\right|^2
+ \sigma^2 |\phi|^2 |S|^2 + \\[2mm]
& + &\left| - \sigma (S \overline{S}) + \kappa \phi^2 + \mu \phi + \Lambda \right|^2\,,\\[2mm]
V_D&=&\ds\frac{g_2^2}{8}\left(H_d^\dagger \sigma_a H_d+H_u^\dagger \sigma_a H_u\right)^2
+\frac{{g'}^2}{8}\left(|H_d|^2-|H_u|^2\right)^2+\\[2mm]
&&+\ds\frac{g^{'2}_1}{2}\left(\tilde{Q}_{H_d} |H_d|^2+\tilde{Q}_{H_u} |H_u|^2+\tilde{Q}_S |S|^2 - \tilde{Q}_S |\overline{S}|^2 \right)^2\,,\\[2mm]
V_{soft}&=&m_{S}^2 |S|^2 + m_{\overline{S}}^2 |\overline{S}|^2 + m_1^2|H_d|^2 + m_2^2|H_u|^2 +
m^2_{\varphi}|\phi|^2+ \biggl[\lambda A_{\lambda} S (H_u H_d) \\[2mm]
&-& \sigma A_{\sigma} \phi (S \overline{S}) + \dfrac{\kappa}{3} A_{\kappa} \phi^3 + B\dfrac{\mu}{2} \phi^2 + \xi \Lambda \phi + h.c.\biggr]\,,
\ea
\label{hd9}
\ee
where $H_d^T=(H_d^0,\,H_d^{-})$, $H_u^T=(H_u^{+},\,H_u^{0})$ and $(H_d H_u)=H_u^{+}H_d^{-}-H_u^{0}H_d^{0}$,
while $\tilde{Q}_{H_d}$, $\tilde{Q}_{H_u}$ and $\tilde{Q}_S$ are effective $U(1)$ charges of $H_d$, $H_u$ and $S$.
At tree--level the Higgs potential in Eq.~(\ref{hd9}) is described by the sum of the first three terms. $V_F$ and $V_D$ contain
$F$--term and $D$--term contributions that do not violate SUSY. The terms in the expression for $V_D$ are proportional to
the $SU(2)_W$, $U(1)_Y$ and $U(1)_{N}$ gauge couplings, i.e. $g_2,\,g'$ and $g'_1$ respectively. The values of the gauge couplings
$g_2$ and $g'$ at the EW scale are well known, whereas the low--energy value of the extra $U(1)_{N}$ coupling $g'_1$ and the effective
$U(1)_{N}$ charges of $H_d$, $H_u$ and $S$ can be calculated assuming gauge coupling unification \cite{King:2005jy}.
The soft SUSY breaking terms are collected in $V_{soft}$. A set of soft SUSY breaking parameters in the tree--level Higgs boson
potential includes the soft masses $m_1^2,\,m_2^2,\,m_{S}^2$,\, $m_{\overline{S}}^2$ and $m^2_{\varphi}$; the trilinear
couplings $A_{\lambda}$, $A_{\sigma}$ and $A_{\kappa}$; the bilinear coupling $B$ and linear coupling $\xi$.
The term $\Delta V$ in Eq.~(\ref{hd9}) is associated with the contribution of loop corrections to the Higgs effective potential.
In SUSY models the most significant contribution to $\Delta V$ comes from the loops involving the top--quark and its superpartners.

At the physical minimum of the scalar potential (\ref{hd9}) the Higgs fields develop VEVs
\be
\ba{c}
<H_d>=\ds\frac{1}{\sqrt{2}}\left(
\begin{array}{c}
v_1\\ 0
\end{array}
\right) , \qquad
<H_u>=\ds\frac{1}{\sqrt{2}}\left(
\begin{array}{c}
0\\ v_2
\end{array}
\right) ,\\[5mm]
<S>=\ds\frac{s_1}{\sqrt{2}},\qquad
<\overline{S}>=\ds\frac{s_2}{\sqrt{2}},\qquad
<\phi>=\ds\frac{\varphi}{\sqrt{2}}\,.
\ea
\label{hd10}
\ee
The equations for the extrema of the Higgs boson potential (\ref{hd9}) in the directions (\ref{hd10}) read:
\be
\ba{rcl}
\dfrac{\partial V}{\partial s_1}&=& m_{S}^2\, s_1 - \dfrac{\lambda A_{\lambda}}{\sqrt{2}} v_1 v_2
- \dfrac{\sigma A_{\sigma}}{\sqrt{2}} \varphi s_2
+ \left(\dfrac{\sigma}{2} s_1 s_2 - \dfrac{\kappa}{2} \varphi^2 - \dfrac{\mu}{\sqrt{2}}\varphi - \Lambda\right)\sigma s_2 \\[2mm]
&+& \dfrac{\sigma^2}{2} \varphi^2 s_1 +
\dfrac{g^{'2}_1}{2}\biggl(\tilde{Q}_{H_d} v_1^2+\tilde{Q}_{H_u} v_2^2+\tilde{Q}_S s_1^2 - \tilde{Q}_S s_2^2 \biggr)\tilde{Q}_S s_1 \\[2mm]
&+& \dfrac{\lambda^2}{2} (v_1^2+v_2^2) s_1 + \dfrac{\partial\Delta V}{\partial s_1}=0\,,\\[2mm]
\dfrac{\partial V}{\partial s_2}&=& m_{\overline{S}}^2\, s_2 - \dfrac{\sigma A_{\sigma}}{\sqrt{2}} \varphi s_1
+ \left(\dfrac{\sigma}{2} s_1 s_2 - \dfrac{\kappa}{2} \varphi^2 - \dfrac{\mu}{\sqrt{2}}\varphi - \Lambda\right)\sigma s_1 \\[2mm]
&+& \dfrac{\sigma^2}{2} \varphi^2 s_2 -
\dfrac{g^{'2}_1}{2}\biggl(\tilde{Q}_{H_d} v_1^2+\tilde{Q}_{H_u} v_2^2+\tilde{Q}_S s_1^2 - \tilde{Q}_S s_2^2 \biggr)\tilde{Q}_S s_2 \\[2mm]
&+& \dfrac{\lambda \sigma}{2} v_1 v_2 \varphi + \dfrac{\partial\Delta V}{\partial s_2}=0\,,\\[2mm]
\dfrac{\partial V}{\partial \varphi}&=& m_{\varphi}^2\, \varphi - \dfrac{\sigma A_{\sigma}}{\sqrt{2}} s_1 s_2 + B\mu \varphi + \sqrt{2}\xi\Lambda
+ \dfrac{\kappa A_{\kappa}}{\sqrt{2}} \varphi^2 + \dfrac{\sigma^2}{2}(s_1^2+s_2^2)\varphi\\[2mm]
&-& 2 \left(\dfrac{\sigma}{2} s_1 s_2 - \dfrac{\kappa}{2} \varphi^2 - \dfrac{\mu}{\sqrt{2}}\varphi - \Lambda\right)\left(\kappa \varphi +\dfrac{\mu}{\sqrt{2}}\right)
+ \dfrac{\lambda \sigma}{2} v_1 v_2 s_2 + \dfrac{\partial\Delta V}{\partial \varphi}=0\,,\\[2mm]
\ds\frac{\partial V}{\partial v_1}&=& m_1^2\, v_1 - \dfrac{\lambda A_{\lambda}}{\sqrt{2}}s_1 v_2 + \dfrac{\lambda \sigma}{2} v_2 s_2 \varphi
+\dfrac{\lambda^2}{2}(v_2^2+s_1^2)v_1+\dfrac{\bar{g}^2}{8}\biggl(v_1^2-v_2^2\biggr)v_1+\\[2mm]
&+&\dfrac{g^{'2}_1}{2}\biggl(\tilde{Q}_{H_d} v_1^2+\tilde{Q}_{H_u} v_2^2+\tilde{Q}_S s_1^2 - \tilde{Q}_S s_2^2 \biggr)\tilde{Q}_{H_d} v_1
+\dfrac{\partial\Delta V}{\partial v_1}=0\,,\\[2mm]
\ds\frac{\partial V}{\partial v_2}&=& m_2^2 v_2-\dfrac{\lambda A_{\lambda}}{\sqrt{2}}s_1 v_1 + \dfrac{\lambda \sigma}{2} v_1 s_2 \varphi
+\dfrac{\lambda^2}{2}(v_1^2+s_1^2)v_2+\dfrac{\bar{g}^2}{8}\biggl(v_2^2-v_1^2\biggr)v_2+\\[2mm]
&&+\ds\frac{g^{'2}_1}{2}\biggl(\tilde{Q}_{H_d}v_1^2+\tilde{Q}_{H_u}v_2^2+\tilde{Q}_S s_1^2 - \tilde{Q}_S s_2^2 \biggr)\tilde{Q}_{H_u} v_2
+ \ds\frac{\partial\Delta V}{\partial v_2}=0\,,
\ea
\label{hd11}
\ee
where $\bar{g}=\sqrt{g_2^2+g'^2}$. Instead of $v_1$, $v_2$, $s_1$ and $s_2$ it is more convenient to use $\tan\beta=v_2/v_1$,
$v=\sqrt{v_1^2+v_2^2}=246\,\mbox{GeV}$, $s=\sqrt{s_1^2+s_2^2}$ and $\tan\theta=s_2/s_1$.

Initially the Higgs sector involves fourteen degrees of freedom. However four of them are massless Goldstone modes.
They are swallowed by the $W^{\pm}$, $Z$ and $Z'$ gauge bosons. The charged $W^{\pm}$ bosons gain masses
via the interaction with the neutral components of the Higgs doublets $H_u$ and $H_d$ just in the same way as in the
MSSM resulting in $M_W=\dfrac{g_2}{2}v$. On the other hand the mechanism of the neutral gauge boson mass
generation differs substantially. Letting $Z'$ and $Z$ states be the gauge bosons associated with $U(1)_{N}$ and
SM--like $Z$ boson the $Z-Z'$ mass squared matrix is given by
\be
M^2_{ZZ'}=\left(
\ba{cc}
\dfrac{\bar{g}^2}{4} v^2  & \Delta^2 \\[2mm]
\Delta^2  & g^{'2}_1v^2\biggl(\tilde{Q}_{H_d}^2\cos^2\beta+\tilde{Q}_{H_u}^2\sin^2\beta\biggr) + g^{'2}_1 \tilde{Q}^2_S s^2
\ea
\right)\,,
\label{hd12}
\ee
where
$$
\Delta^2=\ds\frac{\bar{g}g'_1}{2}v^2\biggl(\tilde{Q}_{H_d}\cos^2\beta-\tilde{Q}_{H_u}\sin^2\beta\biggr)\,.
$$
Since fields $S$ and $\overline{S}$ must acquire large VEVs, i.e. $s_1\simeq s_2 \gg 1\,\mbox{TeV}$ to ensure
that the extra $U(1)_{N}$ gauge boson is heavy enough, the mass of the lightest neutral gauge boson $Z_1$ is very close
to $M_Z=\bar{g}v/2$ whereas the mass of $Z_2$ is set by $g'_1\tilde{Q}_S\, s$.

\subsection{Higgs boson spectrum}

For the analysis of the Higgs boson spectrum we use the equations for the extrema (\ref{hd11}) to express the soft masses
$m_1^2,\,m_2^2,\,m_{S}^2$,\, $m_{\overline{S}}^2$ and $m^2_{\varphi}$ in terms of $s,\,v,\,\varphi,\, \beta,\, \theta$
and other parameters. Because of the electric--charge conservation the charged components of the Higgs doublets are not
mixed with neutral Higgs fields. They form a separate sector whose spectrum is described by a $2\times 2$ mass matrix.
The determinant of this matrix has zero value resulting in the appearance of two Goldstone states
\be
G^{-}=H_d^{-}\cos\beta-H_u^{+*}\sin\beta\,,
\label{hd13}
\ee
which are absorbed into the longitudinal degrees of freedom of the $W^{\pm}$ gauge boson, and two physical states
\be
H^{+}=H_d^{-*}\sin\beta+H_u^{+}\cos\beta
\label{hd14}
\ee
with mass
\be
m^2_{H^{\pm}}=\dfrac{\sqrt{2}\lambda s}{\sin 2\beta}\left(A_{\lambda} \cos\theta - \dfrac{\sigma \varphi}{\sqrt{2}}\sin\theta\right)
-\frac{\lambda^2}{2}v^2+\frac{g_2^2}{4}v^2+\Delta_{\pm}\,,
\label{hd15}
\ee
where $\Delta_{\pm}$ is a contribution of loop corrections to $m^2_{H^{\pm}}$.

The imaginary parts of the neutral components of the Higgs doublets and imaginary parts of the SM singlet fields
$S$, $\overline{S}$ and $\phi$ compose two Goldstone states
\be
\ba{l}
G=\sqrt{2}(\mbox{Im}\,H_d^0\cos\beta - \mbox{Im}\, H_u^0\sin\beta)\,,\\[1mm]
G'=\sqrt{2}(\mbox{Im}\,S \cos\theta - \mbox{Im}\,\overline{S}\sin\theta)\cos\gamma - \sqrt{2}(\mbox{Im}\,H_u^0\cos\beta + \mbox{Im}\,H_d^0\sin\beta)\sin\gamma\,,
\ea
\label{hd16}
\ee
which are swallowed by the $Z$ and $Z'$ bosons, as well as three physical states.
In Eq.~(\ref{hd16}) $\tan\gamma=\dfrac{v}{2s}\sin 2\beta$.
In the field basis $(P_1,\,P_2,\,P_3)$
\be
\ba{l}
P_1=\sqrt{2}(\mbox{Im}\,H_u^0\cos\beta + \mbox{Im}\,H_d^0\sin\beta)\cos\gamma + \sqrt{2}(\mbox{Im}\,S \cos\theta - \mbox{Im}\,\overline{S}\sin\theta)\sin\gamma\,, \\[1mm]
P_2=\sqrt{2}\left(\mbox{Im}\,S\sin\theta + \mbox{Im}\,\overline{S}\cos\theta\right)\,,\\[1mm]
P_3=\sqrt{2} \mbox{Im}\,\phi\,,
\ea
\label{hd17}
\ee
the mass matrix of the CP--odd Higgs sector takes the form
\be
\tilde{M}^2=
\left(
\ba{ccc}
\tilde{M}_{11}^2 & \tilde{M}_{12}^2 & \tilde{M}_{13}^2\\
\tilde{M}_{21}^2 & \tilde{M}_{22}^2 & \tilde{M}_{23}^2\\
\tilde{M}_{31}^2 & \tilde{M}_{32}^2 & \tilde{M}_{33}^2
\ea
\right)\,,
\label{hd18}
\ee
where
\be
\ba{rcl}
\tilde{M}_{11}^2&=&\dfrac{\sqrt{2}\lambda s}{\sin 2\beta \cos^2 \gamma}\left(A_{\lambda} \cos\theta - \dfrac{\sigma \varphi}{\sqrt{2}}\sin\theta\right)+\tilde{\Delta}_{11}\,,\\[3mm]
\tilde{M}_{12}^2&=&\tilde{M}_{21}^2=\dfrac{\lambda v}{\sqrt{2}\cos\gamma}\left(A_{\lambda} \sin\theta + \dfrac{\sigma \varphi}{\sqrt{2}}\cos\theta\right)+\tilde{\Delta}_{12}\,,\\[3mm]
\tilde{M}_{22}^2&=&\dfrac{2\sigma\varphi}{\sin 2\theta}\left(\dfrac{A_{\sigma}}{\sqrt{2}}+\dfrac{\kappa}{2}\varphi+\dfrac{\mu}{\sqrt{2}}+\dfrac{\Lambda}{\varphi}\right)\\[3mm]
&+&\dfrac{\lambda v^2 \sin 2\beta}{\sqrt{2} s \sin 2\theta}\left(A_{\lambda}\sin^3\theta-\dfrac{\sigma\varphi}{\sqrt{2}}\cos^3\theta\right)+\tilde{\Delta}_{22}\,,\\[3mm]
\tilde{M}_{23}^2&=&\tilde{M}_{32}^2=\sigma s \left(\dfrac{A_{\sigma}}{\sqrt{2}}-\kappa \varphi -\dfrac{\mu}{\sqrt{2}} \right) -\dfrac{\lambda\sigma}{4}v^2 \sin 2\beta \cos\theta
+\tilde{\Delta}_{23}\,,\\[3mm]
\tilde{M}_{13}^2&=&\tilde{M}_{31}^2=\dfrac{\lambda\sigma v s}{2\cos \gamma}\sin\theta+\tilde{\Delta}_{13}\,,\\[3mm]
\tilde{M}_{33}^2&=&\dfrac{\sigma s^2}{2\sqrt{2}\varphi} A_{\sigma} \sin 2\theta - 2 B\mu -3\dfrac{\kappa A_{\kappa}}{\sqrt{2}}\varphi
- \sqrt{2}(\xi+\mu)\dfrac{\Lambda}{\varphi} + \sigma\kappa s^2 \sin 2\theta - \dfrac{\kappa\mu}{\sqrt{2}}\varphi \\[3mm]
&-&4\kappa\Lambda + \dfrac{\sigma\mu s^2}{2\sqrt{2}\varphi} \sin 2\theta - \dfrac{\lambda\sigma s}{4\varphi} v^2 \sin\theta \sin 2\beta+\tilde{\Delta}_{33}\,.
\ea
\label{hd19}
\ee
In Eqs.~(\ref{hd19}) $\tilde{\Delta}_{ij}$ correspond to the contribution of loop corrections. Because in the models under consideration $s$ must be much larger
than $v$, $\gamma$ goes to zero. Moreover since in phenomenologically acceptable SUSY models the supersymmetry breaking scale also tends
to be considerably larger than $v$ the mixing between $P_1$ and pseudoscalar states which are mainly the linear superpositions of the imaginary
parts of the SM singlet fields $S$, $\overline{S}$ and $\phi$, i.e. $P_2$ and $P_3$, is somewhat suppressed. In other words the off--diagonal entries
$\tilde{M}_{12}^2,\, \tilde{M}_{13}^2\ll \tilde{M}_{11}^2$ that allows to diagonalize the mass matrix (\ref{hd18})--(\ref{hd19}) analytically.
In particular, the mass of the one of the CP--odd Higgs eigenstate, which is predominantly $P_1$, is set by $\tilde{M}_{11}^2$. As a consequence
this CP--odd state and the charged physical Higgs states are nearly degenerate in general.

The mass matrix (\ref{hd18})--(\ref{hd19}) can be diagonalized by means of unitary transformation that relates the components of the CP--odd
Higgs basis (\ref{hd17}) to the corresponding Higgs mass eigenstates:
\be
\left(
\begin{array}{c}
P_1\\ P_2\\ P_3
\end{array}
\right)=
U
\left(
\begin{array}{c}
A_1 \\ A_2\\ A_3
\end{array}
\right)\,,
\label{hd20}
\ee
where the pseudoscalar mass eigenstates are labeled according to increasing absolute value of mass, with $A_1$ being the lightest
CP--odd Higgs state and $A_3$ the heaviest. In the limit when the couplings $\kappa$, $\mu$ and $\Lambda$, which violate the global $U(1)$
symmetry discussed previously, vanish, the mass of the lightest CP--odd Higgs boson goes to zero. Neglecting all terms proportional to $\lambda v$
and $\tilde{\Delta}_{ij}$ one obtains
\be
\ba{lcl}
m^2_{A_3} & \simeq & \mbox{max} \biggl\{\dfrac{\sqrt{2}\sigma A_{\sigma} \varphi}{\sin 2\theta \cos^2 \delta}\,,\quad
\dfrac{\sqrt{2}\lambda s}{\sin 2\beta}\left(A_{\lambda} \cos\theta - \dfrac{\sigma \varphi}{\sqrt{2}}\sin\theta\right) \biggr\}\,,\\[3mm]
m^2_{A_2} & \simeq & \mbox{min} \biggl\{\dfrac{\sqrt{2}\sigma A_{\sigma} \varphi}{\sin 2\theta \cos^2 \delta}\,,\quad
\dfrac{\sqrt{2}\lambda s}{\sin 2\beta}\left(A_{\lambda} \cos\theta - \dfrac{\sigma \varphi}{\sqrt{2}}\sin\theta\right)\biggr\}\,,\\[3mm]
m^2_{A_1} & \simeq & \cos^2 \delta \left[- 2 B\mu -3\dfrac{\kappa A_{\kappa}}{\sqrt{2}}\varphi
- \sqrt{2}\xi \dfrac{\Lambda}{\varphi} + \dfrac{9}{4} \sigma\kappa s^2 \sin 2\theta \right.\\[3mm]
&&\left. + \sqrt{2} \dfrac{\sigma\mu s^2}{\varphi} \sin 2\theta + \dfrac{\sigma s^2 \Lambda}{2 \varphi^2} \sin 2\theta\right]\,,
\ea
\label{hd21}
\ee
where $\tan\delta \simeq \dfrac{s}{2\varphi}\sin 2\theta$. In this case the lightest CP--odd mass eigenstate is a linear combination of
$P_2$ and $P_3$
 \be
A_1 \simeq  - P_2 \sin\delta + P_3 \cos\delta\,.
\label{hd22}
\ee

The CP--even Higgs sector involves
$\mbox{Re}\,H_d^0$, $\mbox{Re}\,H_u^0$, $\mbox{Re}\,S$, $\mbox{Re}\,\overline{S}$ and $\mbox{Re}\,\phi$.
In the field space basis $(S_1,\,S_2,\,S_3,\,S_4,\,S_5)$, where
\be
\ba{lcl}
\mbox{Re}\,S & = & (S_1\,\cos\theta + S_2\sin\theta + s_1)/\sqrt{2}\,, \\[2mm]
\mbox{Re}\,\overline{S} & = & (-S_1\,\sin\theta + S_2\cos\theta + s_2)/\sqrt{2}\,, \\[2mm]
\mbox{Re}\,\phi & = & (S_3 + \varphi)/\sqrt{2}\,,\\[2mm]
\mbox{Re}\,H_d^0 & = &(S_5 \cos\beta - S_4 \sin\beta+v_1)/\sqrt{2}\,, \\[2mm]
\mbox{Re}\,H_u^0 & = &(S_5 \sin\beta + S_4 \cos\beta+v_2)/\sqrt{2} \\[2mm]
\ea
\label{hd23}
\ee
the mass matrix of the CP--even Higgs sector takes the form
\be
M^2=\left(
\ba{ccccc}
M_{11}^2 & M_{12}^2 & M_{13}^2 & M_{14}^2 & M_{15}^2\\
M_{21}^2 & M_{22}^2 & M_{23}^2 & M_{24}^2 & M_{25}^2\\
M_{31}^2 & M_{32}^2 & M_{33}^2 & M_{34}^2 & M_{35}^2\\
M_{41}^2 & M_{42}^2 & M_{43}^2 & M_{44}^2 & M_{45}^2\\
M_{51}^2 & M_{52}^2 & M_{53}^2 & M_{54}^2 & M_{55}^2\\
\ea
\right)\,.
\label{hd24}
\ee
where
\be
\ba{rcl}
M_{11}^2& = & g^{'2}_1 \tilde{Q}_S^2 s^2 - \dfrac{\sigma^2 s^2}{2} \sin^2 2\theta + \sqrt{2} \sigma A_{\sigma} \varphi \sin 2\theta\\[2mm]
& + &\biggl(\kappa \sigma \varphi^2 + \sqrt{2} \sigma \mu \varphi + 2\sigma \Lambda\biggr) \sin 2\theta
+ \dfrac{\lambda A_{\lambda}}{2\sqrt{2} s} v^2 \cos\theta \sin 2\beta\\[2mm]
& - & \dfrac{\lambda \sigma \varphi}{4 s} v^2 \sin\theta \sin 2\beta + \Delta_{11}\,,\\[2mm]
M_{12}^2& = &M_{21}^2=\dfrac{\sigma^2 s^2}{4} \sin 4\theta - \sqrt{2} \sigma A_{\sigma} \varphi \cos 2\theta \\[2mm]
& - & \biggl(\kappa \sigma \varphi^2 + \sqrt{2} \sigma \mu \varphi + 2\sigma \Lambda\biggr) \cos 2\theta
+ \dfrac{\lambda A_{\lambda}}{2\sqrt{2} s} v^2 \sin \theta \sin 2\beta \\[2mm]
& + & \dfrac{\lambda \sigma \varphi}{4 s} v^2 \cos\theta \sin 2\beta + \Delta_{12}\,,\\[2mm]
M_{13}^2& = & M_{31}^2= \sigma^2 \varphi s \cos 2\theta - \dfrac{\lambda\sigma}{4} v^2\sin\theta\sin 2\beta +  \Delta_{13}\,,\\[2mm]
M_{22}^2& = & \dfrac{\sigma^2 s^2}{2} \sin^2 2\theta + \dfrac{\sqrt{2} \sigma A_{\sigma} \varphi}{\sin 2\theta} \cos^2 2\theta
+ \biggl(\kappa \sigma \varphi^2 + \sqrt{2} \sigma \mu \varphi + 2\sigma \Lambda\biggr) \dfrac{\cos^2 2\theta}{\sin 2\theta}\\[2mm]
&+&\dfrac{\lambda A_{\lambda} v^2}{2\sqrt{2} s \cos\theta} \sin^2 \theta \sin 2\beta
- \dfrac{\lambda \sigma \varphi v^2}{4 s \sin\theta} \cos^2 \theta \sin 2\beta + \Delta_{22}\,,\\[2mm]
M_{23}^2& = & M_{32}^2=-\dfrac{\sigma A_{\sigma}}{\sqrt{2}}s+\sigma^2\varphi s \sin 2\theta
- \sigma s \left(\kappa\varphi+\dfrac{\mu}{\sqrt{2}}\right)\\[2mm]
&+ &\dfrac{\lambda\sigma}{4}v^2 \cos\theta \sin 2\beta + \Delta_{23}\,,\\[2mm]
M_{33}^2& = &\dfrac{\sigma A_{\sigma} s^2}{2\sqrt{2}\varphi} \sin 2\theta - \sqrt{2}\xi\dfrac{\Lambda}{\varphi} + \dfrac{\kappa A_{\kappa}}{\sqrt{2}} \varphi
+\mu\left(\dfrac{\sigma s^2}{2\sqrt{2}\varphi}\sin 2\theta+3\dfrac{\kappa\varphi}{\sqrt{2}}-\dfrac{\sqrt{2}\Lambda}{\varphi}\right) \\[2mm]
&+ &2 \kappa^2 \varphi^2 -\dfrac{\lambda\sigma s}{4\varphi} v^2 \sin\theta \sin 2\beta + \Delta_{33}\,,\\[2mm]
M_{14}^2& = & M_{41}^2=\dfrac{g^{'2}_1}{2}\tilde{Q}_S(\tilde{Q}_{H_u}-\tilde{Q}_{H_d}) s v \sin 2\beta
- \dfrac{\lambda A_{\lambda}}{\sqrt{2}} v \cos\theta \cos 2\beta\\[2mm]
& - & \dfrac{\lambda\sigma}{2}\varphi v \sin\theta \cos 2\beta + \Delta_{14}\,,\\[2mm]
M_{15}^2& = & M_{51}^2=g^{'2}_1 \tilde{Q}_S (\tilde{Q}_{H_d}\cos^2\beta+ \tilde{Q}_{H_u}\sin^2\beta) s v
-\dfrac{\lambda A_{\lambda}}{\sqrt{2}} v \cos\theta \sin 2\beta\,,\\[2mm]
&+ & \lambda^2 v s \cos^2 \theta - \dfrac{\lambda\sigma}{2} \varphi v \sin\theta \sin 2\beta + \Delta_{15}\,,\\[2mm]
M_{24}^2& = & M_{42}^2=\left(-\dfrac{\lambda A_{\lambda}}{\sqrt{2}} v \sin\theta +
\dfrac{\lambda\sigma}{2} \varphi v \cos\theta \right)\cos 2\beta + \Delta_{24}\,,\\[2mm]
M_{25}^2& = & M_{52}^2=\dfrac{\lambda^2}{2} s v \sin 2\theta + \left(-\dfrac{\lambda A_{\lambda}}{\sqrt{2}} v \sin\theta +
\dfrac{\lambda\sigma}{2} \varphi v \cos\theta \right)\sin 2\beta + \Delta_{25}\,,\\[2mm]
M_{34}^2& = & M_{43}^2=\dfrac{\lambda\sigma}{2} s v \sin\theta \cos 2\beta + \Delta_{34}\,,\\[2mm]
M_{35}^2& = & M_{53}^2=\dfrac{\lambda\sigma}{2} s v \sin\theta \sin 2\beta + \Delta_{35}\,,\\[2mm]
M_{44}^2&=&\dfrac{\sqrt{2}\lambda s}{\sin 2\beta}\left(A_{\lambda} \cos\theta - \dfrac{\sigma \varphi}{\sqrt{2}}\sin\theta\right)+
\left(\dfrac{\bar{g}^2}{4}-\dfrac{\lambda^2}{2}\right)v^2 \sin^2 2\beta\\[2mm]
&+&\ds\frac{g^{'2}_1}{4}(\tilde{Q}_{H_u}-\tilde{Q}_{H_d})^2 v^2 \sin^2 2\beta+\Delta_{44}\,,\\[2mm]
\ea
\label{hd25}
\ee
$$
\ba{rcl}
M_{45}^2&=&M_{54}^2=\left(\dfrac{\lambda^2}{4}-\dfrac{\bar{g}^2}{8}\right)v^2 \sin 4\beta
+\dfrac{g^{'2}_1}{2}v^2(\tilde{Q}_{H_u}-\tilde{Q}_{H_d})\times\\[2mm]
&\times &(\tilde{Q}_{H_d}\cos^2\beta+\tilde{Q}_{H_u}\sin^2\beta)\sin 2\beta+\Delta_{45}\, ,\\[2mm]
M_{55}^2&=&\dfrac{\lambda^2}{2}v^2\sin^22\beta+\dfrac{\bar{g}^2}{4}v^2\cos^22\beta+g^{'2}_1 v^2(\tilde{Q}_{H_d}\cos^2\beta+
\tilde{Q}_{H_u}\sin^2\beta)^2+\Delta_{55}\,.
\ea
$$
In Eq.~(\ref{hd25}) $\Delta_{ij}$ represent the contribution of loop corrections. The components of the CP--even Higgs basis (\ref{hd23})
are related to the physical CP--even Higgs eigenstates by virtue of a unitary transformation:
\be
\left(
\begin{array}{c}
S_1\\S_2\\ S_3\\ S_4\\ S_5
\end{array}
\right)=
\tilde{U}
\left(
\begin{array}{c}
h_1 \\ h_2\\ h_3 \\ h_4\\ h_5
\end{array}
\right)\,,
\label{hd26}
\ee
where again the CP--even Higgs eigenstates are labeled according to increasing absolute value of mass,
with $h_1$ being the lightest CP--even Higgs state and $h_5$ the heaviest.

If all SUSY breaking parameters as well as $\lambda s \sim \sigma s \sim \sigma \varphi \sim M_{S}$
are considerably larger than the EW scale, all masses of the CP--even Higgs states except the lightest Higgs
boson mass are determined by the SUSY breaking scale $M_S$. Since no indication of any kind of
physics beyond the SM has been detected at the LHC so far we shall assume here that $M_S$ is of order of
$1\,\mbox{TeV}$ or even higher. Because the minimal eigenvalue of the mass matrix (\ref{hd24})--(\ref{hd25})
is always less than its smallest diagonal element at least one Higgs scalar in the CP--even sector (approximately $S_5$)
remains always light irrespective of the SUSY breaking scale, i.e. $m^2_{h_1}\lesssim M_{55}^2$ like in the MSSM
and NMSSM.  In the interactions with other SM particles this state manifests itself as a SM--like Higgs boson
if $M_S \gg M_Z$.

In the limit when $\lambda \sim \sigma \to 0 $  the off--diagonal entries $M_{24}^2$, $M_{25}^2$, $M_{34}^2$
and $M_{35}^2$ of the mass matrix (\ref{hd24})--(\ref{hd25}) become negligibly small. At the same time
according to Eq.~(\ref{hd3}) the diagonal entry $M_{11}^2$, which is set by the mass of the $Z'$ boson, tends
to be substantially larger than $M_S^2$, i.e. $M_{11}^2\simeq M_{Z'}^2 \sim M_S^2/\sigma^2$, whereas
$\cos 2\theta$ almost vanishes in this case. Indeed, combining the first and the second equations for the extrema
(\ref{hd11}) one obtains the following tree-level expression for $\cos 2\theta$
\be
\cos 2\theta \simeq
\dfrac{m_{\overline{S}}^2-m_{S}^2}{m_{\overline{S}}^2+m_S^2+\sigma^2\varphi^2+g^{'2}_1 \tilde{Q}_S^2 s^2}
\sim \dfrac{M_S^2}{M_{Z'}^2}\sim \sigma^2\,,
\label{hd27}
\ee
that becomes vanishingly small when $M_{Z'}\gg M_S$ and/or $\sigma\sim \lambda\to 0$. In this case the  hierarchical structure
of the mass matrix (\ref{hd24})--(\ref{hd25}) implies that the masses of the $Z'$ boson and  heaviest CP--even Higgs particle
associated with $S_1$ are almost degenerate. Thus the heaviest CP--even Higgs state can be integrated out. The masses of
another CP--even state, which is predominantly $S_4$, another CP--odd state, that corresponds to $P_1$, and charged Higgs states
are also almost degenerate in this limit. Assuming that the Higgs state which is mainly $S_4$ is the second heaviest CP--even Higgs
state and neglecting all terms which are proportional to the couplings $\kappa$, $\mu$ and $\Lambda$, that violate global $U(1)$
symmetry, as well as setting $\cos 2\theta=0$ one obtains the following approximate analytic expressions for the tree-level
masses of the three lightest CP--even Higgs bosons:
\be
\ba{lcl}
m_{h_{3,2}}^2 & \simeq & \dfrac{\sigma^2 s^2}{4}\left[1+\dfrac{A_{\sigma}}{\sqrt{2}\sigma \varphi} \pm
\biggl|1 - \dfrac{A_{\sigma}}{\sqrt{2}\sigma \varphi}\biggr|\sqrt{1+16\,\dfrac{\varphi^2}{s^2}}\right]\,,\\[2mm]
m_{h_{1}}^2 & \simeq & \dfrac{\bar{g}^2}{4}v^2\cos^22\beta\simeq M_{Z}^2 \cos^2 2\beta\,.
\ea
\label{hd28}
\ee
Here it is worth pointing out that in the scenario under consideration the tree--level expression for the SM--like Higgs mass
$m_{h_{1}}^2$ is basically the same as in the MSSM.



\section{Nonstandard Higgs decays}

The presence of light pseudoscalar Higgs state in the particle spectrum can result in the nonstandard decays of the lightest
CP--even Higgs boson in the $U(1)$ extensions of the MSSM under consideration. Expanding the Higgs potential  (\ref{hd9}) 
about its physical minimum one can obtain the trilinear coupling that describes the interaction of the lightest CP--even Higgs 
scalar with the lightest Higgs pseudoscalar states. At the tree level, the corresponding part of the Lagrangian can be written as
\begin{equation}
\begin{array}{l}
\mathcal{L}_{h_1 A_1 A_1}=-G h_1 A_1 A_1\,,
\end{array}
\label{hd29}
\end{equation}
where
\begin{equation}
\begin{array}{l}
G=\tilde{U}_{51}\left\{U_{11}^2 \left[\dfrac{\lambda^2}{4}v\cos^2\gamma (1+\cos^2 2\beta) + \dfrac{\lambda^2}{2} v \sin^2 \gamma \cos^2 \theta
-\dfrac{\bar{g}^2}{8} v \cos^2 \gamma \cos^2 2\beta \right.\right.\\[1mm]
\left.\left.+\dfrac{1}{2}\left(\dfrac{\lambda A_{\lambda}}{\sqrt{2}} \cos\theta -
\dfrac{\lambda \sigma}{2} \varphi \sin\theta\right)\sin 2\gamma+
\dfrac{g_1^{'2}}{2}v\left(\tilde{Q}_{H_d}\cos^2\beta+\tilde{Q}_{H_u}\sin^2\beta\right)\times\right.\right.\\[1mm]
\left.\left.\times\left(\tilde{Q}_{H_d}\sin^2\beta \cos^2\gamma +
\tilde{Q}_{H_u}\cos^2\beta \cos^2\gamma +\tilde{Q}_{S}\sin^2\gamma \cos 2\theta\right)\right]\right.\\[1mm]
\left.+ U_{11} U_{21} \left[\dfrac{\lambda^2}{2} v \sin 2\theta \sin \gamma + g_1^{'2} \tilde{Q}_S v \left(\tilde{Q}_{H_d}\cos^2\beta
+\tilde{Q}_{H_u}\sin^2\beta\right) \sin\gamma \sin 2\theta \right.\right.\\[1mm]
\left.\left. + \left(\dfrac{\lambda A_{\lambda}}{\sqrt{2}}\sin\theta + \dfrac{\lambda \sigma}{2} \varphi \cos\theta\right) \cos\gamma\right]
+ \dfrac{\lambda\sigma}{2} \sin\theta\, U_{11} U_{31} (s\cos\gamma + v \sin 2\beta \sin\gamma)\right.\\[1mm]
\left. + U_{21}^2\left[\dfrac{\lambda^2}{2}v\sin^2\theta - \dfrac{g_1^{'2}}{2} \tilde{Q}_S v \cos 2\theta
\left(\tilde{Q}_{H_d}\cos^2\beta+\tilde{Q}_{H_u}\sin^2\beta\right)\right] \right.\\[1mm]
\left.- \dfrac{\lambda\sigma}{2} v \sin 2\beta \cos\theta\, U_{21} U_{31} \right\}
+\tilde{U}_{41}\left\{U_{11}^2 \left[\left(-\dfrac{\lambda^2}{8}+\dfrac{\bar{g}^2}{16}\right)v\cos^2\gamma \sin 4\beta\right.\right.\\[1mm]
\left.\left.+\dfrac{g_1^{'2}}{4}v\sin 2\beta (\tilde{Q}_{H_u}-\tilde{Q}_{H_d})\left(\tilde{Q}_{H_d}\sin^2\beta \cos^2\gamma +
\tilde{Q}_{H_u}\cos^2\beta \cos^2\gamma \right.\right.\right.\\[1mm]
\left.\left.\left.+\tilde{Q}_{S}\sin^2\gamma \cos 2\theta\right)\right]+\dfrac{g_1^{'2}}{2}\tilde{Q}_{S} (\tilde{Q}_{H_u}-\tilde{Q}_{H_d})
v\sin 2\beta \sin\gamma \sin 2\theta\, U_{11} U_{21} \right.\\[1mm]
\left.+ \dfrac{\lambda\sigma}{2}v\cos 2\beta \sin\gamma \sin\theta\, U_{11} U_{31} - \dfrac{g_1^{'2}}{4}\tilde{Q}_{S} (\tilde{Q}_{H_u}-\tilde{Q}_{H_d})
v\sin 2\beta \cos 2\theta\, U_{21}^2 \right.\\[1mm]
\left.- \dfrac{\lambda\sigma}{2}v\cos 2\beta \cos\theta\, U_{21} U_{31}\right\}
+\tilde{U}_{31}\left\{U_{11}^2 \left[-\dfrac{\lambda\sigma}{4}s\sin\theta \sin 2\beta \cos^2\gamma + \dfrac{\sigma^2}{2}\varphi\sin^2 \gamma  \right.\right.\\[1mm]
\end{array}
\label{hd30}
\end{equation}
$$
\begin{array}{l}
\left.\left.-\dfrac{\lambda\sigma}{4} v \sin 2\gamma \sin\theta
- \dfrac{\sigma}{2}\sin 2\theta \sin^2\gamma\left(\dfrac{A_{\sigma}}{\sqrt{2}}+\kappa\varphi
+\dfrac{\mu}{\sqrt{2}}\right)\right] \right.\\[1mm]
\left.+ U_{11} U_{21} \left[\dfrac{\lambda\sigma}{2}v\cos\theta\cos\gamma
+\sigma\left(\dfrac{A_{\sigma}}{\sqrt{2}}+\kappa\varphi+\dfrac{\mu}{\sqrt{2}}\right)\sin\gamma\cos 2\theta\right] \right.\\[1mm]
\left.+U_{21}^2 \left[\dfrac{\sigma^2}{2}\varphi+\dfrac{\sigma}{2}\left(\dfrac{A_{\sigma}}{\sqrt{2}}+\kappa\varphi
+\dfrac{\mu}{\sqrt{2}}\right)\sin 2\theta\right]-\sigma\kappa s\, U_{21} U_{31}\right.\\[1mm]
\left.+\kappa U_{31}^2 \left(\kappa\varphi
+\dfrac{\mu}{\sqrt{2}} - \dfrac{A_{\kappa}}{\sqrt{2}}\right)
\right\}+\tilde{U}_{21}\left\{U_{11}^2 \left[-\dfrac{\lambda\sigma}{4}\varphi \sin 2\beta \cos^2\gamma \cos\theta \right.\right.\\[1mm]
\left.\left.+ \dfrac{\lambda^2}{4}s\cos^2 \gamma \sin 2\theta + \dfrac{\lambda A_{\lambda}}{2\sqrt{2}}\sin 2\beta \cos^2 \gamma \sin\theta
+\dfrac{\sigma^2}{4}s \sin^2 \gamma \sin 2\theta\right]\right.\\[1mm]
\left. + U_{11} U_{31}\left[\dfrac{\lambda\sigma}{2}v\cos\gamma\cos\theta+\sigma\left(\dfrac{A_{\sigma}}{\sqrt{2}}-\kappa\varphi
-\dfrac{\mu}{\sqrt{2}}\right) \sin\gamma \cos 2\theta \right]\right.\\[1mm]
\left. + \dfrac{\sigma^2}{4}s\sin 2\theta\, U_{21}^2+\sigma\left(\dfrac{A_{\sigma}}{\sqrt{2}}-\kappa\varphi
-\dfrac{\mu}{\sqrt{2}}\right)\sin 2\theta\, U_{21} U_{31}+\dfrac{\sigma}{2}\left(\sigma s \sin 2\theta + \kappa s\right) U_{31}^2 \right\}\\[1mm]
+\tilde{U}_{11}\left\{U_{11}^2 \left[\dfrac{\lambda\sigma}{4}\varphi \sin 2\beta \cos^2\gamma \sin\theta
+ \dfrac{\lambda^2}{2}s\cos^2 \gamma \cos^2 \theta + \dfrac{\lambda A_{\lambda}}{2\sqrt{2}}\sin 2\beta \cos^2 \gamma \cos\theta\right.\right.\\[1mm]
\left.\left.+\dfrac{g_1^{'2}}{2} \tilde{Q}_{S} s \left(\tilde{Q}_{H_d}\sin^2\beta \cos^2\gamma +
\tilde{Q}_{H_u}\cos^2\beta \cos^2\gamma + \tilde{Q}_{S}\sin^2\gamma \cos 2\theta\right)\right]\right.\\[1mm]
\left.+\left[-\dfrac{\lambda\sigma}{2}v\cos\gamma\sin\theta+\sigma\left(\kappa\varphi+\dfrac{\mu}{\sqrt{2}}
-\dfrac{A_{\sigma}}{\sqrt{2}}\right)\sin\gamma \sin 2\theta \right] U_{11} U_{31} \right.\\[1mm]
\left. + \left( g_1^{'2} \tilde{Q}_{S}^2 - \dfrac{\sigma^2}{2} \right) s \sin\gamma \sin 2\theta\, U_{11} U_{21} +
\left[ \dfrac{\sigma^2}{2} - \dfrac{g_1^{'2}}{2} \tilde{Q}_{S}^2\right] s \cos 2\theta\, U_{21}^2\right.\\[1mm]
\left.+ \sigma\left(\dfrac{A_{\sigma}}{\sqrt{2}}-\kappa\varphi-\dfrac{\mu}{\sqrt{2}}\right)\cos 2\theta\, U_{21} U_{31}
+\dfrac{\sigma^2}{2}s\cos 2\theta\, U_{31}^2\right\}\,.
\end{array}
$$
The interaction Lagrangian (\ref{hd29})--(\ref{hd30}) gives rise to decays of the lightest CP--even Higgs boson into
a pair of lightest pseudoscalars if these sates are lighter than $60\,\mbox{GeV}$. The corresponding partial decay width is given by
\begin{equation}
\Gamma(h_1\to A_1 A_1)=\dfrac{G^2}{8\pi m_{h_1}}\sqrt{1-\dfrac{4 m_{A_1}^2}{m_{h_1}^2}}\,.
\label{31}
\end{equation}
%In order to ensure that the branching ratio associated with these exotic Higgs decays is less than $30\%$ the coupling $G$
%should be relatively small, i.e. $G\lesssim 2-2.5\,\mbox{GeV}$.





%Following the traditional notations we
%define the normalised $R$--couplings as: $g_{ZZh_i}=R_{ZZi}\times$
%SM coupling; $g_{ZAh_i}=\ds\frac{\bar{g}}{2}R_{ZAi}$.  The
%absolute values of all $R$--couplings vary from zero to unity.


















% Thus in Superstring
%inspired $E_6$ models the $\mu$--problem is solved in a similar way to
%the Next--to--Minimal Supersymmetric Standard Model (NMSSM)
%\cite{nmssm}, but without the accompanying problems of singlet
%tadpoles or domain walls \cite{14}.

%cESSM-final.tex


%cESSM_LHC-Signatures

%The extra $U(1)_{N}$ gauge symmetry survives to low energies and
%forbids a bilinear term $\mu {H}_d {H}_u$ in the superpotential but
%allows the interaction $\lambda S H_d H_u$. At the electroweak (EW)
%scale, the scalar component of the SM singlet superfield $S$ acquires
%a non-zero VEV, $\langle S \rangle=s/\sqrt{2}$, breaking $U(1)_N$ and
%yielding an effective $\mu=\lambda s/\sqrt{2}$ term.  Thus the $\mu$
%problem in the E$_6$SSM is solved in a similar way to the
%next-to-minimal supersymmetric standard model (NMSSM)~\cite{nmssm}, but
%without the accompanying problems of singlet tadpoles or domain
%walls.

%ce6ssm125Higgs.tex

%The ATLAS and CMS Collaborations have recently presented the first
%indication for a Higgs boson with a mass about
%$125$~GeV, consistent with the allowed window of Higgs masses
%$125 \pm 3$ GeV \cite{ATLAS:2012ae,Chatrchyan:2012tx}.
%%A well known observation is that while the LHC can in principle
%%exclude a Standard Model (SM) Higgs boson, it can only discover a
%%SM-like Higgs boson, which could for example correspond to a SUSY
%%Higgs boson near the decoupling region\cite{Haber}.

%In the Exceptional Supersymmetric Standard Model (E$_6$SSM) \cite{King:2005jy,King:2005my},
%the spectrum of the MSSM is extended to fill out three complete 27-dimensional representations
%of the gauge group E$_6$ which is broken at the unification scale down to the SM gauge group
%plus one additional gauged $U(1)_N$ symmetry at low energies under which
%right-handed neutrinos are neutral, allowing them to get large masses.
%Each $27$-plet contains one generation of ordinary matter; singlet fields, $S_i$; up and down type
%Higgs doublets, $H_{u,i}$ and $H_{d,i}$; charged $\pm 1/3$ coloured exotics $D_i$, $\bar{D}_i$.
%The extra matter ensures anomaly cancellation, however
%the model also contains two extra SU(2) doublets, $H'$ and $\bar{H}'$, which are required for
%gauge coupling unification \cite{King:2007uj}. To evade rapid proton decay either a $Z_2^B$ or
%$Z_2^L$ symmetry is introduced and to evade large Flavour Changing Neutral Currents an
%approximate $Z_2^H$ symmetry is introduced which ensures that only the third family of Higgs
%$H_{u,3}$ and $H_{d,3}$ couple to fermions and get vacuum expectation values (VEVs).
%Similarly only the third family singlet $S_3$ gets a VEV, $\langle S_3 \rangle = s/\sqrt{2}$,
%which is responsible for the effective $\mu$ term and D-fermion mass.
%The first and second families of Higgs and singlets which do not get VEVs are called ``inert''.













\newpage
\begin{thebibliography}{999}

\bibitem{1}
The ATLAS Collaboration,  ATLAS-CONF-2013-017.

\bibitem{2}
The CMS Collaboration,
% “Search for supersymmetry with the razor variables at CMS”,
CMS-PAS-EXO-12-061.

\bibitem{Kolda:1995iw}
C.~F.~Kolda and S.~P.~Martin,
%``Low-energy supersymmetry with D term contributions to scalar masses,''
Phys.\ Rev.\ D {\bf 53} (1996) 3871
[hep-ph/9503445].

\bibitem{Peccei:1977hh}
R.~D.~Peccei and H.~R.~Quinn,
%``CP Conservation In The Presence Of Instantons,''
Phys.\ Rev.\ Lett.\  {\bf 38} (1977) 1440;
%``Constraints Imposed By CP Conservation In The Presence Of Instantons,''
Phys.\ Rev.\ D {\bf 16} (1977) 1791.

\bibitem{axion}
F.~Wilczek,
%``Problem Of Strong P And T Invariance In The Presence Of Instantons,''
Phys.\ Rev.\ Lett.\  {\bf 40} (1978) 279.

\bibitem{Langacker:2008yv}
P.~Langacker,
%``The Physics of Heavy $Z^\prime$ Gauge Bosons,''
Rev.\ Mod.\ Phys.\  {\bf 81} (2009) 1199
[arXiv:0801.1345 [hep-ph]].

\bibitem{E6-review}
J.L.~Hewett, T.G.~Rizzo, Phys. Rept. {\bf 183} (1989) 193.

\bibitem{E6}
J.~F.~Gunion, H.~E.~Haber, G.~L.~Kane, S.~Dawson,
``The Higgs Hunter's Guide'' (Westview Press, 2000) [Erratum arXiv:hep-ph/9302272];
P.~Binetruy, S.~Dawson, I.~Hinchliffe, M.~Sher,
%``Phenomenologically Viable Models From Superstrings?,''
Nucl. \ Phys. \ B {\bf 273} (1986) 501;
%%CITATION = NUPHA,B273,501;%%
J.~R.~Ellis, K.~Enqvist, D.~V.~Nanopoulos, F.~Zwirner,
%``Observables In Low-Energy Superstring Models,''
Mod.\ Phys.\ Lett.\  A {\bf 1} (1986) 57.
%%CITATION = MPLAE,A1,57;%%
L.~E.~Ibanez, J.~Mas,
%``Low-Energy Supergravity And Superstring Inspired Models,''
Nucl. \ Phys. \ B {\bf 286} (1987) 107;
%%CITATION = NUPHA,B286,107;%%
J.~F.~Gunion, L.~Roszkowski, H.~E.~Haber,
%``Z-Prime Mass Limits, Masses And Couplings Of Higgs Bosons, And Z-Prime Decays
%In An E(6) Superstring Based Model,''
Phys. \ Lett. \ B {\bf 189} (1987) 409;
%%CITATION = PHLTA,B189,409;%%
H.~E.~Haber, M.~Sher,
%``Higgs Mass Bound In E(6) Based Supersymmetric Theories,''
Phys. \ Rev. \ D {\bf 35} (1987) 2206;
%%CITATION = PHRVA,D35,2206;%%
J.~R.~Ellis, D.~V.~Nanopoulos, S.~T.~Petcov, F.~Zwirner,
%``Gauginos And Higgs Particles In Superstring Models,''
Nucl. \ Phys. \ B {\bf 283} (1987) 93;
%%CITATION = NUPHA,B283,93;%%
M.~Drees,
%``Comment On 'Higgs Boson Mass Bound In E(6) Based Supersymmetric Theories. ',''
Phys. \ Rev. \ D {\bf 35} (1987) 2910;
%%CITATION = PHRVA,D35,2910;%%
J.~F.~Gunion, L.~Roszkowski, H.~E.~Haber,
%``Z-prime MASS LIMITS, MASSES AND COUPLINGS OF HIGGS BOSONS, AND Z-prime
%DECAYS IN AN E(6) SUPERSTRING BASED MODEL,''
Phys.\ Lett.\  B {\bf 189} (1987) 409;
  %%CITATION = PHLTA,B189,409;%%
H.~Baer, D.~Dicus, M.~Drees, X.~Tata,
%``Higgs Boson Signals In Superstring Inspired Models At Hadron
%Supercolliders,''
Phys. \ Rev. \ D {\bf 36} (1987) 1363;
%%CITATION = PHRVA,D36,1363;%%
J.~F.~Gunion, L.~Roszkowski, H.~E.~Haber,
%``Production And Detection Of The Higgs Bosons Of The Simplest E(6) Based Gauge
%Theory,''
Phys. \ Rev. \ D {\bf 38} (1988) 105.
%%CITATION = PHRVA,D38,105;%%


%\cite{Accomando:2010fz}
\bibitem{Accomando:2010fz}
E.~Accomando, A.~Belyaev, L.~Fedeli, S.~F.~King, C.~Shepherd-Themistocleous,
%``Z' physics with early LHC data,''
Phys.\ Rev.\  D {\bf 83} (2011) 075012
[arXiv:1010.6058 [hep-ph]].

%\cite{Kang:2007ib}
\bibitem{Kang:2007ib}
J.~Kang, P.~Langacker, B.~D.~Nelson,
%``Theory and Phenomenology of Exotic Isosinglet Quarks and Squarks,''
Phys.\ Rev.\  D {\bf 77} (2008) 035003
[arXiv:0708.2701 [hep-ph]].
%%CITATION = PHRVA,D77,035003;%%

%\cite{Langacker:1998tc}
\bibitem{Langacker:1998tc}
P.~Langacker, J.~Wang,
%``U(1)' symmetry breaking in supersymmetric E(6) models,''
Phys.\ Rev.\  D {\bf 58} (1998) 115010.


\bibitem{Cvetic:1997ky}
M.~Cveti$\check{\rm c}$, P.~Langacker, Phys.\ Rev.\ D {\bf 54} (1996) 3570;
M.~Cveti$\check{\rm c}$, P.~Langacker, Mod.\ Phys.\ Lett.\ A {\bf 11} (1996) 1247;
M.~Cvetic, D.~A.~Demir, J.~R.~Espinosa, L.~L.~Everett and P.~Langacker,
%``Electroweak breaking and the mu problem in supergravity models with an additional U(1),''
Phys.\ Rev.\ D {\bf 56} (1997) 2861 [Erratum-ibid.\ D {\bf 58} (1998) 119905].


%\cite{Suematsu:1994qm}
\bibitem{Suematsu:1994qm}
D.~Suematsu, Y.~Yamagishi,
%``Radiative symmetry breaking in a supersymmetric model with an extra U(1),''
Int.\ J.\ Mod.\ Phys.\  A {\bf 10} (1995) 4521.

%\cite{Keith:1997zb}
\bibitem{Keith:1997zb}
E.~Keith, E.~Ma,
%``Generic consequences of a supersymmetric U(1) gauge factor at the TeV
%scale,''
Phys.\ Rev.\  D {\bf 56} (1997) 7155.

%\cite{Daikoku:2000ep}
\bibitem{Daikoku:2000ep}
Y.~Daikoku, D.~Suematsu,
%``Mass bound of the lightest neutral Higgs scalar in the extra U(1)
%models,''
Phys.\ Rev.\  D {\bf 62} (2000) 095006.


\bibitem{Kang:2004ix}
J.~H.~Kang, P.~Langacker, T.~J.~Li,
%``Neutrino masses in supersymmetric SU(3)C x SU(2)L x U(1)Y x U(1)'
%models,''
Phys.\ Rev.\  D {\bf 71} (2005) 015012.

\bibitem{Ma:1995xk}
E.~Ma,
%``Neutrino masses in an extended gauge model with E(6) particle content,''
Phys.\ Lett.\  B {\bf 380} (1996) 286.

\bibitem{Hambye:2000bn}
T.~Hambye, E.~Ma, M.~Raidal, U.~Sarkar,
%``Allowable low-energy E(6) subgroups from leptogenesis,''
Phys.\ Lett.\  B {\bf 512} (2001) 373.


\bibitem{King:2008qb}
S.~F.~King, R.~Luo, D.~J.~Miller, R.~Nevzorov,
%``Leptogenesis in the Exceptional Supersymmetric Standard Model: flavour
%dependent lepton asymmetries,''
JHEP {\bf 0812} (2008) 042.

\bibitem{baryogen}
E.~Ma, M.~Raidal, J.\ Phys.\ G {\bf 28} (2002) 95;
J.~Kang, P.~Langacker, T.-J.~Li, T.~Liu, Phys.\ Rev.\ Lett.\ {\bf 94} (2005) 061801.

\bibitem{g-2}
J.~A.~Grifols, J.~Sola, A.~Mendez,
%``CONTRIBUTION TO THE MUON ANOMALY FROM SUPERSTRING INSPIRED MODELS,''
Phys.\ Rev.\ Lett.\  {\bf 57} (1986) 2348;
D.~A.~Morris,
%``POTENTIALLY LARGE CONTRIBUTIONS TO THE MUON ANOMALOUS MAGNETIC MOMENT FROM
%WEAK ISOSINGLET SQUARKS IN E(6) SUPERSTRING MODELS,''
Phys.\ Rev.\  D {\bf 37} (1988) 2012.

\bibitem{Suematsu:1997tv}
D.~Suematsu,
%``Effect on the electron EDM due to abelian gauginos in SUSY extra U(1)
%models,''
Mod.\ Phys.\ Lett.\  A {\bf 12} (1997) 1709.

%\cite{GutierrezRodriguez:2006hb}
\bibitem{GutierrezRodriguez:2006hb}
A.~Gutierrez-Rodriguez, M.~A.~Hernandez-Ruiz and M.~A.~Perez,
%``Limits on the Electromagnetic and Weak Dipole Moments of the Tau-Lepton in
%E(6) Superstring Models,''
Int.\ J.\ Mod.\ Phys.\  A {\bf 22}, 3493 (2007)
[arXiv:hep-ph/0611235].
%%CITATION = IMPAE,A22,3493;%%

\bibitem{Suematsu:1997qt}
D.~Suematsu,
%``mu --> e gamma in supersymmetric multi U(1) models with an abelian  gaugino
%mixing,''
Phys.\ Lett.\  B {\bf 416} (1998) 108.

%\cite{Ham:2008fx}
\bibitem{Ham:2008fx}
S.~W.~Ham, J.~O.~Im, E.~J.~Yoo and S.~K.~Oh,
%``Higgs bosons of a supersymmetric $E_6$ model at the Large Hadron
%Collider,''
JHEP {\bf 0812}, 017 (2008)
[arXiv:0810.4194 [hep-ph]].
%%CITATION = JHEPA,0812,017;%%

\bibitem{Suematsu:1997au}
D.~Suematsu,
%``Neutralino decay in the mu problem solvable extra U(1) models,''
Phys.\ Rev.\  D {\bf 57} (1998) 1738.

\bibitem{Keith:1996fv}
E.~Keith, E.~Ma,
%``Efficacious Extra U(1) Factor for the Supersymmetric Standard Model,''
Phys.\ Rev.\  D {\bf 54} (1996) 3587.

\bibitem{Hesselbach:2001ri}
S.~Hesselbach, F.~Franke, H.~Fraas,
%``Neutralinos in E(6) inspired supersymmetric U(1)' models,''
Eur.\ Phys.\ J.\  C {\bf 23} (2002) 149;
V.~Barger, P.~Langacker, H.~S.~Lee,
%``Lightest neutralino in extensions of the MSSM,''
Phys.\ Lett.\  B {\bf 630} (2005) 85;
S.~Y.~Choi, H.~E.~Haber, J.~Kalinowski, P.~M.~Zerwas,
%``The neutralino sector in the U(1)-extended supersymmetric standard model,''
Nucl.\ Phys.\  B {\bf 778} (2007) 85;
V.~Barger, P.~Langacker, I.~Lewis, M.~McCaskey, G.~Shaughnessy and B.~Yencho,
%``Recoil detection of the lightest neutralino in MSSM singlet extensions,''
Phys.\ Rev.\  D {\bf 75} (2007) 115002.

\bibitem{Gherghetta:1996yr}
T.~Gherghetta, T.~A.~Kaeding, G.~L.~Kane,
%``Supersymmetric contributions to the decay of an extra $Z$ boson,''
Phys.\ Rev.\ D {\bf 57} (1998) 3178
[hep-ph/9701343].

\bibitem{E6neutralino-higgs}
V.~Barger, P.~Langacker, G.~Shaughnessy,
%``TeV physics and the Planck scale,''
New J.\ Phys.\  {\bf 9} (2007) 333.

%\cite{Asano:2008ju}
\bibitem{Asano:2008ju}
M.~Asano, T.~Kikuchi and S.~G.~Kim,
%``D-term assisted Anomaly Mediation in E6 motivated models,''
arXiv:0807.5084 [hep-ph].
%%CITATION = ARXIV:0807.5084;%%

%\cite{Stech:2008wd}
\bibitem{Stech:2008wd}
B.~Stech and Z.~Tavartkiladze,
%``Generation Symmetry and E_6 Unification,''
Phys.\ Rev.\  D {\bf 77}, 076009 (2008)
[arXiv:0802.0894 [hep-ph]].

\bibitem{King:2005jy}
S.~F.~King, S.~Moretti, R.~Nevzorov,
%``Theory and phenomenology of an exceptional supersymmetric standard
%model,''
Phys.\ Rev.\  D {\bf 73} (2006) 035009.

\bibitem{King:2005my}
S.~F.~King, S.~Moretti, R.~Nevzorov,
%``Exceptional supersymmetric standard model,''
Phys.\ Lett.\  B {\bf 634} (2006) 278.

\bibitem{Accomando:2006ga}
S.~F.~King, S.~Moretti, R.~Nevzorov,
%``Spectrum of Higgs particles in the ESSM,''
arXiv:hep-ph/0601269;
S. Kraml {\it et al.} (eds.), {\it Workshop on CP studies and
non-standard Higgs physics}, CERN--2006--009, hep-ph/0608079;
S.~F.~King, S.~Moretti, R.~Nevzorov,
%``E(6)SSM,''
AIP Conf.\ Proc.\  {\bf 881} (2007) 138;
%[arXiv:hep-ph/0610002].

\bibitem{E6-higgs}
V.~Barger, P.~Langacker, H.~S.~Lee, G.~Shaughnessy,
%``Higgs Sector in Extensions of the MSSM,''
Phys.\ Rev.\  D {\bf 73} (2006) 115010.

%\cite{King:2007uj}
\bibitem{King:2007uj}
S.~F.~King, S.~Moretti, R.~Nevzorov,
%``Gauge coupling unification in the exceptional supersymmetric standard
%model,''
Phys.\ Lett.\  B {\bf 650} (2007) 57
[arXiv:hep-ph/0701064].

\bibitem{Howl:2007zi}
R.~Howl, S.~F.~King,
%``Minimal E(6) Supersymmetric Standard Model,''
JHEP {\bf 0801} (2008) 030
[arXiv:0708.1451 [hep-ph]];
P.~Athron, J.~P.~Hall, R.~Howl, S.~F.~King, D.~J.~Miller, S.~Moretti, R.~Nevzorov,
%``Aspects of the exceptional supersymmetric standard model,''
Nucl.\ Phys.\ Proc.\ Suppl.\  {\bf 200-202} (2010) 120.


\bibitem{Nevzorov:2013tta}
R.~Nevzorov, S.~Pakvasa,
%``Exotic Higgs decays in the $E_6$ inspired SUSY models,''
Phys.\ Lett.\ B {\bf 728} (2014) 210
[arXiv:1308.1021 [hep-ph]].

\bibitem{Hall:2010ix}
J.~P.~Hall, S.~F.~King, R.~Nevzorov, S.~Pakvasa, M.~Sher,
%``Novel Higgs Decays and Dark Matter in the E(6)SSM,''
Phys.\ Rev.\  D {\bf 83} (2011) 075013
[arXiv:1012.5114 [hep-ph]];
J.~P.~Hall, S.~F.~King, R.~Nevzorov, S.~Pakvasa, M.~Sher,
%``Nonstandard Higgs decays in the $E_6SSM$,''
arXiv:1012.5365 [hep-ph];
J.~P.~Hall, S.~F.~King, R.~Nevzorov, S.~Pakvasa, M.~Sher,
%``Nonstandard Higgs Decays and Dark Matter in the E6SSM,''
arXiv:1109.4972 [hep-ph].


\bibitem{cE6SSM}
P.~Athron, S.~F.~King, D.~J.~Miller, S.~Moretti, R.~Nevzorov,
%``The Constrained E$_6$SSM,''
arXiv:0810.0617 [hep-ph];
P.~Athron, S.~F.~King, D.~J.~Miller, S.~Moretti, R.~Nevzorov,
%``Predictions of the Constrained Exceptional Supersymmetric Standard Model,''
Phys.\ Lett.\  B {\bf 681} (2009) 448
[arXiv:0901.1192 [hep-ph]];
P.~Athron, S.~F.~King, D.~J.~Miller, S.~Moretti, R.~Nevzorov,
%``The Constrained Exceptional Supersymmetric Standard Model,''
Phys.\ Rev.\  D {\bf 80} (2009) 035009
[arXiv:0904.2169 [hep-ph]];
P.~Athron, S.~F.~King, D.~J.~Miller, S.~Moretti, R.~Nevzorov,
%``LHC Signatures of the Constrained Exceptional Supersymmetric Standard
%Model,''
Phys.\ Rev.\ D {\bf 84} (2011) 055006
[arXiv:1102.4363 [hep-ph]];
P.~Athron, S.~F.~King, D.~J.~Miller, S.~Moretti, R.~Nevzorov,
%``Constrained Exceptional Supersymmetric Standard Model with a Higgs Near 125 GeV,''
Phys.\ Rev.\ D {\bf 86} (2012) 095003
[arXiv:1206.5028 [hep-ph]].

%\cite{Athron:2013ipa}
\bibitem{Athron:2013ipa}
  P.~Athron, M.~Binjonaid and S.~F.~King,
  %``Fine Tuning in the Constrained Exceptional Supersymmetric Standard Model,''
  Phys.\ Rev.\ D {\bf 87}, no. 11, 115023 (2013)
  [arXiv:1302.5291 [hep-ph]].
  %%CITATION = ARXIV:1302.5291;%%
  %10 citations counted in INSPIRE as of 23 May 2014

\bibitem{Athron:2012pw}
P.~Athron, D.~Stockinger,  A.~Voigt,
%``Threshold Corrections in the Exceptional Supersymmetric Standard Model,''
Phys.\ Rev.\ D {\bf 86} (2012) 095012
[arXiv:1209.1470 [hep-ph]].
%\cite{Miller:2012vn}

\bibitem{Miller:2012vn}
  D.~J.~Miller, A.~P.~Morais and P.~N.~Pandita,
  %``Constraining Grand Unification using first and second generation sfermions,''
  Phys.\ Rev.\ D {\bf 87}, 015007 (2013)
  [arXiv:1208.5906 [hep-ph]].
  %%CITATION = ARXIV:1208.5906;%%
  %3 citations counted in INSPIRE as of 23 May 2014

\bibitem{Sperling:2013eva}
M.~Sperling, D.~Stöckinger, A.~Voigt,
%``Renormalization of vacuum expectation values in spontaneously broken gauge theories,''
JHEP {\bf 1307} (2013) 132
[arXiv:1305.1548 [hep-ph]];
  M.~Sperling, D.~Stöckinger and A.~Voigt,
  %``Renormalization of vacuum expectation values in spontaneously broken gauge theories: Two-loop results,''
  JHEP {\bf 1401}, 068 (2014)
  [arXiv:1310.7629 [hep-ph]].
  %%CITATION = ARXIV:1310.7629;%%
  %5 citations counted in INSPIRE as of 23 May 2014

\bibitem{nevzorov}
R.~Nevzorov,
%``E6 Inspired SUSY Models with Exact Custodial Symmetry,''
Phys.\ Rev.\ D {\bf 87} (2013) 015029
[arXiv:1205.5967 [hep-ph]].



\bibitem{Chang:2008cw}
S.~Chang, R.~Dermisek, J.~F.~Gunion, N.~Weiner,
%``Nonstandard Higgs Boson Decays,''
Ann.\ Rev.\ Nucl.\ Part.\ Sci.\  {\bf 58} (2008) 75
[arXiv:0801.4554 [hep-ph]];
R.~Dermisek,
%``Unusual Higgs or Supersymmetry from Natural Electroweak Symmetry
%Breaking,''
Mod.\ Phys.\ Lett.\  A {\bf 24} (2009) 1631
[arXiv:0907.0297 [hep-ph]].

\bibitem{review-nmssm}
M.~Maniatis,
%``The Next-to-Minimal Supersymmetric extension of the Standard Model reviewed,''
Int.\ J.\ Mod.\ Phys.\ A {\bf 25} (2010) 3505
[arXiv:0906.0777 [hep-ph]];
U.~Ellwanger, C.~Hugonie, A.~M.~Teixeira,
%``The Next-to-Minimal Supersymmetric Standard Model,''
Phys.\ Rept.\  {\bf 496} (2010) 1
[arXiv:0910.1785 [hep-ph]];
U.~Ellwanger,
%``Higgs Bosons in the Next-to-Minimal Supersymmetric Standard Model at the LHC,''
Eur.\ Phys.\ J.\ C {\bf 71} (2011) 1782
[arXiv:1108.0157 [hep-ph]].

\bibitem{ArkaniHamed:2006mb}
N.~Arkani-Hamed, A.~Delgado and G.~F.~Giudice,
%``The Well-tempered neutralino,''
Nucl.\ Phys.\ B {\bf 741} (2006) 108
[hep-ph/0601041];
G.~Chalons, M.~J.~Dolan and C.~McCabe,
%``Neutralino dark matter and the Fermi gamma-ray lines,''
JCAP {\bf 1302} (2013) 016
[arXiv:1211.5154 [hep-ph]].

\bibitem{43} S.~Wolfram, Phys.\ Lett.\ B {\bf 82} (1979) 65;
C.~B.~Dover, T.~K.~Gaisser, G.~Steigman, Phys.\ Rev.\ Lett. {\bf 42} (1979) 1117.

\bibitem{42} J.~Rich, M.~Spiro, J.~Lloyd--Owen, Phys.\ Rept.  {\bf 151} (1987) 239;
P.~F.~Smith, Contemp.\ Phys.  {\bf 29} (1988) 159; T.~K.~Hemmick et al. ,
Phys.\ Rev.\  D {\bf 41} (1990) 2074.

\bibitem{45}
G.~F.~Giudice, A.~Masiero, Phys.\ Lett.\ B {\bf 206} (1988) 480;
J.~A.~Casas, C.~Mu$\tilde{\rm n}$oz, Phys.\ Lett.\ B {\bf 306} (1993) 288.

\bibitem{King:2007uj}
S.~F.~King, S.~Moretti, R.~Nevzorov,
%``Gauge coupling unification in the exceptional supersymmetric standard
%model,''
Phys.\ Lett.\  B {\bf 650} (2007) 57
[arXiv:hep-ph/0701064].

\bibitem{Hesselbach:2007te}
S.~Hesselbach, D.~J.~.~Miller, G.~Moortgat-Pick, R.~Nevzorov and M.~Trusov,
%``Theoretical upper bound on the mass of the LSP in the MNSSM,''
Phys.\ Lett.\  B {\bf 662} (2008) 199
[arXiv:0712.2001 [hep-ph]];
S.~Hesselbach, D.~J.~.~Miller, G.~Moortgat-Pick, R.~Nevzorov and M.~Trusov,
%``The lightest neutralino in the MNSSM,''
arXiv:0710.2550 [hep-ph];
S.~Hesselbach, G.~Moortgat-Pick, D.~J.~Miller, 2, R.~Nevzorov and M.~Trusov,
%``Lightest Neutralino Mass in the MNSSM,''
arXiv:0810.0511 [hep-ph].

\bibitem{Frere:1996gb}
J.~M.~Frere, R.~B.~Nevzorov and M.~I.~Vysotsky,
%``Stimulated neutrino conversion and bounds on neutrino magnetic moments,''
Phys.\ Lett.\  B {\bf 394}, 127 (1997)
[arXiv:hep-ph/9608266].

\bibitem{Nevzorov:2013ixa}
R.~Nevzorov,
%``Quasi-fixed point scenarios and the Higgs mass in the E6 inspired SUSY models,''
Phys.\ Rev.\ D {\bf 89} (2014) 055010
[arXiv:1309.4738 [hep-ph]].

\bibitem{Nevzorov:2001vj}
R.~B.~Nevzorov and M.~A.~Trusov,
%``Infrared quasifixed solutions in the NMSSM,''
Phys.\ Atom.\ Nucl.\  {\bf 64} (2001) 1299
[Yad.\ Fiz.\  {\bf 64} (2001) 1375]
[hep-ph/0110363].

\bibitem{Miller:2005qua}
D. J. Miller, S. Moretti, R. Nevzorov, {\it Proceedings to the 18th International
Workshop on High-Energy Physics and Quantum Field Theory (QFTHEP 2004)},
ed. by M.N. Dubinin, V.I. Savrin, Moscow, Moscow State Univ., 2004. p. 212; hep-ph/0501139

\bibitem{Nevzorov:2004ge}
D.~J.~.~Miller, R.~Nevzorov and P.~M.~Zerwas,
%``The Higgs sector of the next-to-minimal supersymmetric standard model,''
Nucl.\ Phys.\  B {\bf 681} (2004) 3
[arXiv:hep-ph/0304049];
R. Nevzorov, D. J.  Miller, {\it Proceedings to the 7th Workshop "What comes beyond
the Standard Model"}, ed.  by N. S. Mankoc-Borstnik, H. B. Nielsen, C. D. Froggatt,
D. Lukman, DMFA--Zaloznistvo, Ljubljana, 2004, p. 107; hep-ph/0411275.

\bibitem{Nevzorov:2001um}
P.~A.~Kovalenko, R.~B.~Nevzorov and K.~A.~Ter-Martirosian,
%``Masses of Higgs bosons in supersymmetric theories,''
Phys.\ Atom.\ Nucl.\  {\bf 61} (1998) 812
[Yad.\ Fiz.\  {\bf 61} (1998) 898];
R.~B.~Nevzorov and M.~A.~Trusov,
%``Particle spectrum in the modified NMSSM in the strong Yukawa coupling
%limit. (In Russian),''
J.\ Exp.\ Theor.\ Phys.\  {\bf 91} (2000) 1079
[Zh.\ Eksp.\ Teor.\ Fiz.\  {\bf 91} (2000) 1251]
[arXiv:hep-ph/0106351];
R.~B.~Nevzorov, K.~A.~Ter-Martirosyan and M.~A.~Trusov,
%``Higgs bosons in the simplest SUSY models,''
Phys.\ Atom.\ Nucl.\  {\bf 65} (2002) 285
[Yad.\ Fiz.\  {\bf 65} (2002) 311]
[arXiv:hep-ph/0105178].

\end{thebibliography}

\end{document}
